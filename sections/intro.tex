\section{Introduction}

High energy particles are created in nuclear processes including man-made 
devices, cosmic radiation, and natural interactions. 
There is a wide variety of particles with varying properties. 
Detecting and characterizing them is crucial for a broad range of applications 
including nuclear security \cite{vetter2018gamma}, nuclear reactor and stockpile imaging \cite{beaumont2015high}, 
medical imaging \cite{gonzalez2021evolution}, 
archeology \cite{menichelli2007scintillating,ryzewski2013neutron}, 
and astronomy \cite{schonfelder1984imaging}. 
A particularly useful property of these particles is that many of them can 
penetrate through dense materials and therefore allow us to image through 
barriers and inside solid structures. 
The same penetrating properties, however, make it challenging to build a camera 
that can perform imaging or vision using high energy particles. 
A detector needs to have a large volume and density for a particle to interact. 
At the same time, we would like to know the location and shape of the interaction if it happens. 
One common approach to achieve this is by using a scintillator.

A scintillator is a radiation detector that converts ionizing radiation into 
visible light.
It provides mass for an incident particle to be absorbed and detected with a photosensor.
While a particle is propagating through a scintillator, it may deposit energy and 
cause a ``scintillation event". 
In the case of gamma-ray radiation, a gamma-ray collides with an electron in the 
scintillator and causes the electron to recoil over a random walk.
Scintillation photons are emitted isotropically from the electron's path over a 
decay time.
The photons propagate out of the scintillator and are captured by a sensor.
Measuring the event, such as its position, time, and energy deposition, is 
performed with the obtained signal.
These measurements are then used in various downstream tasks to characterize 
the radiation source.

Scintillation light collection methods consider both scintillator hardware design  
and sensor characteristics to optimize event measurements for a given task.
In this work, we focus on the design of measuring events in a monolithic 
scintillator with a camera.
A drawback of this design is its low light collection compared to that from 
coupling a sensor directly to the scintillator's surface.
The reasons are that the event must be far enough from the lens to be imaged, and 
the lens adds a restrictive aperture and reduced range of acceptance angles for 
incident light.
This drawback has a smaller impact in tasks where measurements of individual 
particles or gamma-rays are not needed.
For example, in certain radiography tasks, the overall glow from ionizing 
radiation being absorbed in a scintillator after being attenuated in an object is 
used to form an image of the object.
In this case, increasing total exposure time can increase the signal to 
noise ratio (SNR), and low-speed cameras (CMOS, CCD, etc.) are adequate \cite{pleinert1997design,baker2014scintillator,adams2017gamma,balasubramanian2022x,gustschin2024event}.
On the other hand, measuring individual events requires high-speed, single-photon 
sensors due high particle incidence rates and the finite number of photons emitted 
by an event.
In this case, increasing exposure time does not increase light collection per 
event and decreases SNR.
Therefore, maximizing light throughput is crucial for this kind of task.
The ``Compton camera" \cite{kataoka2013handy, hosokoshi2019development, llosa2019sipm, parajuli2022development, kim2024comprehensive} 
and neutron scatter camera \cite{mascarenhas2006development,marleau2007advances,mascarenhas2009results,weinfurther2018model} 
are examples of devices that measure events from individual gamma-rays or neutrons.
These devices measure double-interaction events and use the kinematics of 
gamma-ray or neutron scattering to perform backprojection for localizing the 
radiation source. % \cite{haefner2015filtered, xu2006filtered, wilderman1998list}.
Recently, single-photon-avalanche-diode (SPAD) cameras have been of interest for 
measuring events for their high spatial and temporal resolution at the sensor \cite{bocchieri2024scintillation}.
SPAD cameras could potentially make high-resolution measurements of individual 
events in a thick scintillator volume and perform advanced radiation imaging 
techniques, but they are limited by low levels of light collection.

Reflective surfaces and scintillator geometry have been considered for increasing 
light collection in various designs.
%Retroreflectors have been shown to improve light collection in parallelpiped and 
%pyramidal frustum scintillators and a position-sensitive photomultiplier (PSPMT) 
%sensor \cite{ros2014retroreflector}. 
Retroreflectors have been shown to improve light collection along with spatial 
and energy resolution in monolithic scintillators.  \cite{mcelroy2002use,heemskerk2009micro,ros2014retroreflector,gonzalez2017performance}.
Specular and diffuse reflectors have been tested in the context of improving 
reflection models for simulation purposes \cite{janecek2010simulating,roncali2017integrated,trigila2021optimization}.
These studies use designs where a sensor is coupled to the scintillator's surface.
To the best of our knowledge, reflective surfaces have not been reported for 
collecting additional light in camera designs.
Adding diffuse surfaces will reflect light into the camera but obfuscate the 
event's image and cause a loss in spatial information.
Retroreflector surfaces better preserve spatial information but can no more than 
double the amount of captured light.
Retroreflectors also add blur that may degrade spatial resolution.
Current camera designs only capture light emitted directly to the camera,
which is a small fraction of all the light emitted from an event 
\cite{bocchieri2024scintillation, yamamoto2023development, d2021novel, gao2023novel, losko2021new,adams2017gamma,balasubramanian2022x,gustschin2024event}.
The goal of this work is to increase light collection by capturing additional 
light from an event that is emitted away from the camera.
The challenge lies in doing so without losing the event's spatial information.
%Another option for increasing light collection would be to surround the 
%scintillator with multiple cameras.
%However, the extra camera hardware increases overall cost and complexity and may 
%cause incident particles to scatter before reaching the scintillator.

We propose to use a kaleidoscopic scintillator geometry with specular surfaces to 
redirect light into the camera.
A kaleidoscope consists of planar mirrors oriented like a pyramid frustum.
%that produce multiple images of an object from different viewpoints.
Light from inside the kaleidoscope reflects one or more times before 
propagating out of the pyramid's open base and into a camera.
The resulting image contains direct light from the event and indirect light from 
the event's mirror reflections. 
Mirror reflections appear as events in locations determined by the kaleidoscope's geometry.
Mirrors redirect light and preserve its structure, thus preserving the event's 
spatial information.
The kaleidoscope can increase light collection by multiple factors.
In this work, we demonstrate an improvement in light collection by up to about 
five times that compared to imaging direct light only.
The task is to determine the event's 3D position from the resulting image.

%With sufficient light, complex signatures such as multi-interaction events, 
%particle tracks, and Cherenkov photons could be measured.


Mirrors and kaleidoscopic designs have been used for stereo vision \cite{nene1998stereo, gluckman1999planar, gluckman2002rectified} 
and multi-view 3D reconstructions \cite{reshetouski2011three, ahn2021kaleidoscopic, ahn2023neural, takahashi2021structure, bangay2004kaleidoscope, mitsumoto19923}
of an extended object using one picture from a single camera.
A ``light-trap" design consists of mirrors oriented such that light entering the 
trap reaches nearly every position inside the trap \cite{xu2018trapping}.
Time-of-flight is used to reconstruct a 3D object inside the trap.
In fact, the pyramid-shaped light-trap was found to provide the best object 
coverage, which is the same shape as the kaleidoscope we use in this work.
Here, we apply the kaleidoscope in the photon-starved setting of scintillation 
event imaging and form a radiation detection problem as a computer vision problem.


In this paper, we make the following contributions: 
\begin{itemize}
\item A new scintillator design for increasing light collection in a camera while 
preserving the event's spatial information.
\item Theory for modeling light from an event in a kaleidoscopic scintillator that 
arrives at the camera sensor.
\item An algorithm to estimate an event's location in a kaleidoscopic scintillator. 
The algorithm is validated on experimental data captured with a SPAD camera and a 
gamma-ray source.
\item \textcolor{red}{Is the captured dataset a contribution?}
\end{itemize}
An overview of the method is illustrated in \cref{fig:teaser}.


\ifthenelse{\boolean{figs_in_text}}{

\begin{figure}
\centering
\includegraphics[width=\linewidth]{teaser_pipeline.jpg}
\caption{\textbf{Overview of image capture and event localization method.}
An image is composed of light that is emitted from the scintillation event and 
reaches the camera either directly or after reflecting off mirrors of a 
kaleidoscopic scintillator.
The locations of an event's mirror reflections are known for a given event 
location and kaleidoscope geometry.
This relationship between the event and mirror reflections is embedded in a 
Gaussian mixture model whose likelihood is maximized to estimate the event's location.
} 
\label{fig:teaser}
\end{figure}

}{}

