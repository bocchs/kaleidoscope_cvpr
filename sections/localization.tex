\section{Event localization algorithm}

We adopt a 2D Gaussian 
\begin{equation}
\mathcal{N}(\bm{t};\bm{\mu},\sigma^2)=\frac{1}{2\pi\sigma^2}\exp\left({-\frac{(\bm{t}-\bm{\mu})^T(\bm{t}-\bm{\mu})}{2\sigma^2}}\right) 
\end{equation}
as the camera's point spread function and assume a circular Gaussian with 
covariance matrix $\Sigma=\sigma^2 I_{2\times2}$. 
$\sigma=ac$, where $a$ is a proportionality constant determined by the 
optical configuration, and $c$ is the circle of confusion diameter in \cref{eqn:circ_of_conf}.
$\bm{t}$ is a 2D coordinate on the sensor plane.

An event or mirror reflection located at $\bm{p_k}=(x_k,y_k,z_{ck})$ is modeled as 
a point source of light, so its image on the sensor consists of photon arrivals 
spatially distributed over a 2D Gaussian with mean 
\begin{equation} \label{eqn:mu}
\bm{\mu_k}=\left[ -\frac{S_2}{z_{ck}}x_k, -\frac{S_2}{z_{ck}}y_k \right]
\end{equation}
and standard deviation 
\begin{equation} \label{eqn:stdev}
\sigma=aA\frac{S_2}{S_1}\frac{|S_1-z_{ck}|}{z_{ck}}
\end{equation}
where \cref{eqn:mu} is derived from perspective projection.
Each photon arrival is one sample from this Gaussian.
%\textcolor{red}{
%In this paper, we flip the pixels of a frame so that photons arrive on the same 
%side of the sensor as the event (e.g. an event in the positive $x$-$y$ quadrant 
%produces an image on the sensor in the positive $x$-$y$ quadrant).
%This is the same as removing the negative signs in \cref{eqn:mu}.}

We model an image of an event in a kaleidoscopic scintillator as a Gaussian 
mixture model (GMM), where each component in the GMM corresponds to the event or a 
mirror reflection.
The complete-data likelihood function, $L$, for an image with $N$ photons, 
one event, and $K$ mirror reflections is
\begin{equation}
L(\bm{\theta};\bm{t},\bm{z})=\prod_{i=1}^N \prod_{k=0}^K \left[ \pi_k \mathcal{N}(\bm{t_i};\bm{\mu_k},\sigma_k^2) \right]^{\mathbbm{1}{(z_i=k)}}
\end{equation}
where $\bm{\theta}=(\bm{\mu},\bm{\sigma},\bm{\pi})$ are model parameters, 
and $\pi_k=P(z_i=k)$ is the prior distribution that photon $i$ comes from 
component $k$.
$\bm{t}=(\bm{t_1}, \bm{t_2}, ..., \bm{t_N})$ are the 2D coordinates of each of 
$N$ photon arrivals on the sensor and $\bm{z}=(z_1, z_2, ..., z_N)$ are the latent 
variables of which component in the GMM that a photon belongs to.
We apply a density-based weighting scheme to photon samples to minimize the 
influence of sparsely distributed dark counts.
The weighted complete-data likelihood function, $L_w$, is 
\begin{align}
L_w(\bm{\theta};\bm{t},\bm{z}) &= \prod_{i=1}^N \prod_{k=0}^K \left[ \pi_k \mathcal{N}(\bm{t_i};\bm{\mu_k},\frac{1}{w_i}\sigma_k^2) \right]^{\mathbbm{1}{(z_i=k)}} \\
&= \prod_{i=1}^N \prod_{k=0}^K \left[ \pi_k \frac{w_i}{2\pi{\sigma_k}^2} \text{exp}\left( -\frac{w_i}{2{\sigma_k}^2} (\bm{t_i}-\bm{\mu_k})^T(\bm{t_i}-\bm{\mu_k}) \right) \right]^{\mathbbm{1}{(z_i=k)}}
\end{align}
where
\begin{align} \label{eqn:weights}
w_i = \sum_{j \in S_i^q} \text{exp} \left( -\frac{||\bm{t_i}-\bm{t_j}||_2^2}{\nu} \right)
\end{align}
is the weight assigned to photon sample $i$, $S_i^q$ is the set of $q$ 
nearest neighbors of photon $i$, and $\nu$ is a positive scalar.
The expected value of the weighted complete-data log-likelihood, $Q$, is
\begin{equation} \label{eqn:Q_eqn}
\begin{aligned}
Q & = E_{\bm{z}|\bm{t}}\left[\log L_w(\bm{\theta};\bm{t},\bm{z})\right] \\ & = \sum_i \sum_k r_{ik} \left[ \text{log}(\pi_k) + \text{log}(w_i) - \text{log}(2\pi{\sigma_k}^2) - \frac{w_i{\sigma_k}^{-2}}{2}(\bm{t_i}-\bm{\mu_k})^T(\bm{t_i}-\bm{\mu_k}) \right]
\end{aligned}
\end{equation}
where 
\begin{equation} \label{eqn:r_ik}
\begin{aligned}
r_{ik} & = E_{\bm{z}|\bm{t}}[\mathbbm{1}{(z_i=k)}] \\ & = \frac{\pi_k \mathcal{N}(\bm{t_i};\bm{\mu_k},{\sigma_k^2})}{\sum_{k'=0}^K \pi_{k'} \mathcal{N}(\bm{t_i};\bm{\mu_{k'}},{\sigma_{k'}^2})}
\end{aligned}
\end{equation}
gives the posterior distribution of $\bm{z}$.
$r_{ik}$ is the probability that photon $i$ comes from event $k$, given 
the current parameter values.

Truncations can be incorporated in setting the value of $r_{ik}$. 
If photon $i$ at location $\bm{t_i}$ on the sensor lies in the truncation zone 
belonging to mirror reflection $k$, then the probability that photon 
$i$ belongs to mirror reflection $k$ is 0. 
In this case, $r_{ik}=0$, and the corresponding term in the 
summation in the denominator of \cref{eqn:r_ik} is 0 when solving for a different $r_{ik}$.

The location of each mirror reflection at $\bm{p_k}$ for $k=1...K$ generated from 
an event at $\bm{p_0}$ is known based on the kaleidoscope's geometry.
For any mirror reflection $k$, each coordinate in $(x_k,y_k,z_{ck})$ is a linear 
combination of the event's $(x_0,y_0,z_{c0})$ coordinates based on the mirror's 
reflection transformation.
All $\bm{p_k}$'s can be written in terms of $(x_0,y_0,z_{c0})$ using 
\cref{eqn:ref_trans}, and all $\bm{\mu_k}$'s and $\sigma_k$'s can be written in 
terms of $(x_0,y_0,z_{c0})$ using \cref{eqn:mu,eqn:stdev}.
We can reduce the number of free parameters from $O(K)$ 
to $O(1)$ by rewriting each $\bm{\mu_k}$ and $\sigma_k$ in terms of 
$(x_0,y_0,z_{c0})$.
Thus, we obtain a GMM where each component is constrained to $\bm{p_0}$, and 
$\bm{\theta}=(x_0,y_0,z_{c0},\bm{\pi})$.
This results in an optimization problem for estimating $\bm{p_0}$ that captures 
the global information of the event and all mirror reflections:
\begin{equation} \label{eqn:opti_prob}
\argmax_{x_0,y_0,z_{c0},\bm{\pi}} Q
\end{equation}
We optimize this using the EM algorithm.
$Q$ parameterized in terms of $\bm{p_0}$ is derived in Supplementary \cref*{sec:like,sec:weighted_like}.

One or more images of mirror reflections may be missing due to truncations, 
depending on the event's location.
Therefore, determining which mirror reflections are present is required to compute $Q$.
We run an initialization procedure to determine the presence of mirror reflections 
and to initialize the event's estimated location for the EM algorithm.

We assume a square pyramid kaleidoscope geometry, up to single-order reflections, 
and the presence of at least two mirror reflections in an image.
Each side of the scintillator's square surface is parallel to each respective side 
of the sensor.
Mirror reflections will appear along either the $x$ or $y$ axes from the event.
In the initialization procedure, centroids $\bm{\mu_k}^0$ are obtained using 
weighted KMeans.
We obtain combinations of subsets of centroids, compute the standard deviation of 
a subset's coordinates along the $x$ and $y$ axes to determine which centroids are 
horizontally or vertically distributed, and classify centroids based on their 
relative positioning. 
Centroids are classified as either the event, or $+x$, $-x$, $+y$, or $-y$ mirror reflections. 
A set of possible event locations that spans the depth of the scintillator is 
computed using the event's centroid and \cref{eqn:mu}.
The number of mirror reflections, which reflections are present, and the 
initialization point for $\bm{p_0}$ are those that correspond to the 
highest value of $Q$ out of this set over 3, 4, and 5 clusters in KMeans.
The number of clusters corresponds to the number of mirror reflections plus the event.
This initialization procedure is described in \cref{alg:init}.
$\bm{\pi}$ is initialized to the uniform distribution.


\begin{algorithm}
\caption{\textbf{Initialization procedure.}} \label{alg:init}
\DontPrintSemicolon
\SetKwInOut{Parameter}{Arguments}
\SetKwFunction{KMeans}{KMeans}
\SetKwFunction{ClassifyCentroids}{ClassifyCentroids}
\SetKwFunction{PossibleEventLocations}{PossibleEventLocations}
\SetKwFunction{Q}{Q}
\SetKwComment{Comment}{// }{}
%\begin{algorithmic}
\Parameter{photon locations $\bm{t}$, photon weights $\bm{w}$, lens to sensor distance $S_2$}
\KwOut{event initialization location $\bm{p_{init}}$, number of mirror reflections $K_{out}$, mirror reflection classification $M_{out}$}
$Q_{max} = -\infty$\;
\For{K=3:5}{
  $\bm{\mu_k}^0 \leftarrow \KMeans(\bm{t}, \bm{w}, K)$\;
  $M \leftarrow \ClassifyCentroids(\bm{\mu_k}^0)$\;
  $S \leftarrow \PossibleEventLocations(\bm{\mu_{event}}^0, S_2)$ \Comment{\cref{eqn:mu}}
  \For{$\bm{\tilde{p_0}} \in S$}{
    $Q_{\bm{\tilde{p_0}}} \leftarrow \Q(\bm{\tilde{p_0}})$\;
    \If{$Q_{\bm{\tilde{p_0}}} > Q_{max}$}{
      $Q_{max} \leftarrow Q_{\bm{\tilde{p_0}}}$\;
      $\bm{p_{init}} \leftarrow \bm{\tilde{p_0}}$\;
      $M_{out} \leftarrow M$\;
      $K_{out} \leftarrow K-1$\;
    }
  }
}
\end{algorithm}

During the E-step, $r_{ik}$ is updated using \cref{eqn:r_ik} and the current 
values of $\bm{p_0}$ and $\bm{\pi}$.
During the M-step, $\bm{p_0}$ is updated by optimizing 
$Q$ with gradient ascent, 
and $\bm{\pi}$ is updated using $\pi_k=\frac{1}{N}\sum_{i=1}^N r_{ik}$.
Gradients are derived in Supplementary \cref*{sec:like,sec:weighted_like}.
In both the E and M steps, terms that correspond to a mirror reflection $k$ are 
included in the computation only if that mirror reflection is present in the image.
If truncation boundaries are known, then partial image truncations 
can be incorporated into the algorithm by zeroing $r_{ik}$ for photon
$i$ that lies in mirror $k$'s truncation zone.
We do not have accurate knowledge of truncation boundaries in experimental data,
so we only determine the presence of mirror reflections rather than partial 
truncations in the experiments.

