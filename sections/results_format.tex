\section{Results}
\iffalse
We test a kaleidoscopic scintillator with a square pyramid geometry (four side-surfaces).
An image is assumed to contain the event and at most one mirror reflection per 
side-surface of the kaleidoscope.
Each event and mirror reflection appears as a defocused point source of light in 
the image.
Light from mirror reflections may be truncated along lines on the sensor 
corresponding to mirrors' edges or completely removed, as described by the theory 
introduced in the ``Methods" section.
Our algorithm determines which mirror reflections are present in an image and 
estimates the event's location by running the EM algorithm on a Gaussian mixture 
model (GMM) parameterized in terms of the event's location.
\fi

Results are summarized below. See the ``Methods" section for more information on 
the theory of kaleidoscopic light collection, simulated image generation, 
algorithm, and experiments.

A simulated image overlaid with each mirror reflection's acceptance zone of where 
photons can arrive is shown in \cref{fig:trunc_examples}. 
This figure is used to validate the theory on kaleidoscopic light collection and 
image truncations. 

Selected experimental images overlaid with the algorithm's estimated Gaussian 
components are shown in \cref{fig:example_figures} to validate the kaleidoscopic 
model and the presence of mirror reflections.

The distances between mean and individual estimated event locations in groups of 
experimental images of removed mirror reflections are reported in \cref{fig:crossval_error}.
Statistics on the distances in the single- and double-removed mirror reflection 
group over different ranges of photon counts are reported in \cref{tab:crossval_thresh_error}.
Distances between estimated event locations using the regular and fixed 
initialization methods, as well as the number of EM steps taken, are shown in \cref{fig:convergence}.
These results are for validating that the algorithm is estimating the event's 
location and converging in optimization.


\ifthenelse{\boolean{figs_in_text}}{

\begin{figure}
\centering
\includegraphics[width=\linewidth]{trunc_examples.pdf}
\caption{\textbf{Simulated kaleidoscopic image with theoretical acceptance zones.} 
Acceptance zones are derived for each mirror reflection and overlaid on the image 
for the (a) +$x$, (b) +$y$, (c) -$x$, and (d) -$y$ mirror reflections.} 
\label{fig:trunc_examples}
\end{figure}



\begin{figure*}
\centering
\includegraphics[width=\linewidth]{example_figures.pdf}
\caption{\textbf{Selected experimental images.} Experimental images overlaid with the algorithm's estimated Gaussians.
Each dashed red circle is centered on the Gaussian component's mean. 
The inner and outer circles are one and two standard deviations in radius, respectively.
Pixels with a photon are enlarged with a $3 \times 3$ filter for visualization purposes.
} 
\label{fig:example_figures}
\end{figure*}



\begin{figure*}
\centering
\includegraphics[width=\linewidth]{crossval_error.pdf}
\caption{\textbf{Agreement in event location measurements.} Histogram of distances between mean estimated event location and each image's estimated event location after mirror reflection removals. 
(a) Distances including both single and double mirror reflection removals. Mean $\pm$ stdev: 0.24 $\pm$ 0.34 mm, 24,761 distances.
(b) Distances including single mirror reflection removal. Mean $\pm$ stdev: 0.16 $\pm$ 0.28 mm, 11,255 distances.
(c) Distances including double mirror reflection removal. Mean $\pm$ stdev: 0.27 $\pm$ 0.35 mm, 15,757 distances.
} 
\label{fig:crossval_error}
\end{figure*}

\begin{figure*}
\centering
\includegraphics[width=\linewidth]{fixed_init_results.pdf}
\caption{\textbf{Optimization convergence.}
(a) Distances in estimated event locations between the regular and fixed initialization methods. Mean $\pm$ stdev: $0.38 \pm 0.80$ mm, 4,351 events.
(b) Nummber of steps taken in the EM algorithm from the regular initialization point. Mean $\pm$ stdev: $9.3 \pm 14.2$ steps, 4,351 events.
(c) Nummber of steps taken in the EM algorithm from the fixed initialization point. Mean $\pm$ stdev: $13.5 \pm 14.9$ steps, 4,351 events.
} 
\label{fig:convergence}
\end{figure*}


\begin{table}[h!]
\centering
\begin{tabular}{|c|ccccc|}
\hline
Counts in a test image & 60 - 79 & 80 - 99 & 100 - 119 & 120 - 139 & 140 - 159 \\
\hline
Mean (mm)                & 0.28 & 0.26 & 0.19 & 0.19 & 0.15 \\
Standard dev.  (mm)      & 0.36 & 0.35 & 0.30 & 0.29 & 0.21  \\
Number of distances    & 9,317  & 7,524  & 4,774  & 2,354  & 792 \\
\hline
\end{tabular}
\caption{\textbf{Agreement in event location measurements.} Mean and standard deviation of distances between mean estimated 
event location and each image's estimated event location after both single and 
double mirror reflection removals reported over different ranges of counts in a 
test image.}
\label{tab:crossval_thresh_error}
\end{table}

}{}


