\section{Discussion}

One kaleidoscopic geometry is tested in this work with a wide opening angle at the 
apex so that imaging second-order reflections is rare or impossible. 
However, the kaleidoscope is configurable in terms of number of surfaces, opening 
angle, and size.
For example, choosing a smaller opening angle will increase the number of mirror 
reflections and light collection at the cost of a smaller scintillator volume and 
higher-complexity images with less distance between mirror reflections.
Dividing the scintillator into more surfaces would increase the number of mirror 
reflections but also the amount of image truncation.
Overall, the kaleidoscopic scintillator can increase imaged photon counts by 
multiple factors for increased energy resolution and provide multiple views for 
measuring events with increased accuracy and robustness using a single camera.

\noindent
\textbf{Simulated validation of theoretical image truncations.}
The mirror reflections generated with ray tracing and a thin lens are truncated 
along the theoretically derived acceptance zones, shown in \cref{fig:trunc_examples}. 
This suggests the theoretical truncation model is correct.
In \cref{fig:trunc_examples}, the +$x$, -$x$, and -$y$ mirror reflections are 
partially truncated, while the +$y$ mirror reflection is not truncated.
Obtaining accurate experimental acceptance zones would require camera 
calibration for lens distortions, adapting the scintillator's edges accordingly, 
and precise knowledge of the focal plane.

\noindent
\textbf{Experimental validation of kaleidoscopic image model.}
Various examples where the Gaussian components predicted by the 
algorithm approximately overlap the photon clusters of the event and mirror 
reflections by visual inspection are shown in \cref{fig:example_figures}.
This suggests that mirror reflections are indeed being captured in accordance with 
the proposed model.
Examples also demonstrate the algorithm can correctly determine the number of 
mirror reflections present in the image.
\cref{fig:example_figures}a,b are examples where an event occurred at high and 
low $z_w$-coordinates, respectively.
\cref{fig:example_figures}c,d illustrate cases where a mirror reflection lies 
outside the field of view. 
\cref{fig:example_figures}e,f show examples where mirror reflections would lie in 
the field of view but are completely truncated.
\cref{fig:example_figures}e illustrates a case where the +$y$ mirror reflection is 
completely truncated, and \cref{fig:example_figures}f illustrates a case where the 
-$x$ and -$y$ mirror reflections are completely truncated.
Dark counts in \cref{fig:example_figures} are higher than the median dark count 
rate of 4 pixels per image.
This could be due to several factors, including increased cross talk from higher 
photon collection, fluorescence from other particles or gamma-rays occurring at 
the beginning or end of an image's integration time, imperfect mirror reflections, 
or internal reflections.

Internal reflections may be contributing to background noise in addition to dark counts.
However, internal reflections at the scintillator's base surface could possibly 
appear in an image as another mirror reflection of the event.
An internal reflection has the same effect as adding a mirror to the 
scintillator's base surface, which can be modeled with the theory introduced in 
this paper.
Since internal reflection occurs at a large incidence angle, the 
mirror reflection of an internal reflection will likely exist 
outward from the optical axis away from mirror reflections formed 
without internal reflection. 
The camera's FOV in the experiments is limited so that we do not 
expect to image mirror reflections that follow from internal reflections.
Imaging internal reflections will increase light collection but complicate the 
captured image.

\noindent
\textbf{Experimental validation of localization algorithm.}
The results in \cref{fig:crossval_error} demonstrate an agreement in event 
location measurements among combinations of mirror reflection removals.
This agreement suggests the algorithm is in fact estimating the event's location, 
and that mirror reflections provide robustness in addition to increased photon counts.

The mean distance in event location estimates between regular and fixed 
initialization points is small, shown in \cref{fig:convergence}a.
This indicates that the gradient ascent optimization during the M-step does in 
fact step toward the event location and approaches it.
\cref{fig:convergence}b,c indicate that setting the initialization point using the 
algorithm's initialization proecdure results in convergence with fewer EM steps 
because of its proximity to the event compared to a fixed initialization point. 
Slightly elevated frequency in convergence distances between 1 and 2 mm shown in 
\cref{fig:convergence}a indicates that a fixed initialization can result in 
convergence at a local optimum rather than the event location.

\noindent
\textbf{Limitations and future work.}
A goal of this work is to validate that our proposed localization algorithm is in 
fact locating the event.
We do so by reporting that the algorithm obtains spatially close estimates of an 
event's location over multiple images of the event after removing mirror reflections.
This provides a measure of localization precision.
Performing a more complete characterization of the experimental system, 
including event localization accuracy and resolution, 
would require scanning the scintillator with a collimated gamma-ray source in
order to obtain a form of ground truth event locations.
This work is meant to demonstrate a new radiation detector concept.
Future work may optimize the hardware configuration or extend the algorithm for 
specific downstream tasks and characterize overall performance more thoroughly.
Further investigation of localization accuracy for this specific experimental 
configuration would provide little additional insight at this stage.

While this work considers one event per image, multiple events from a single 
particle or gamma-ray can occur. 
Measuring multi-event cases is needed to perform advanced radiation imaging 
techniques such as in the Compton camera or neutron scatter camera.
Future work may consider how to model images that contain multiple events in a 
kaleidoscopic scintillator.

\textcolor{red}{The calibration procedure resulted in Gaussian mixture model (GMM) components 
whose centroids are close to but not exactly centered on the mirror reflections in 
the calibration image in \cref{fig:calibration}.
Similarly in \cref{fig:example_figures}c, the Gaussian component's centroid is 
slightly offset from the -$y$ mirror reflection.
This is likely due to $Q$'s definition and lens distortion.
An event's location influences both the mean and standard deviation of its image.
$Q$ takes both parameters into account jointly over the event and all mirror 
reflections, so the optimal $Q$ value may not correspond to components in the GMM 
that are all centered on the event or mirror reflections.
Lens distortions may also cause shifts in the event and mirror reflections that 
are not accounted for in the model.
Nonetheless, GMM components are overlaid accurately on the event and mirror 
reflections in \cref{fig:example_figures}a,b,d,e,f with the same calibrated parameters.}

\iffalse
\textcolor{red}{
While this work models one event, multiple events from a single particle or 
gamma-ray can occur. 
For example, a neutron or gamma-ray can scatter one or more times before exiting 
or being absorbed in the scintillator.
Measuring a pair of scatter-absorption events from a single particle forms the 
basis for backprojection to determine the particle's trajectory and the radiation 
source's location.
In general, the ability to measure multiple, simultaneous events allows to 
accurately measure how and where a particle interacted in the scintillator.
Adequate energy, spatial, and time resolutions are required to perform the Compton backprojection.
SPAD camera designs have sufficient spatial and time resolutions, but 
they are limited by low light collection.
Now, the kaleidoscopic scintillator may provide the light collection levels needed 
to perform these advanced techniques with a SPAD camera.
Future work may consider how to model images that contain multiple events in a 
kaleidoscopic scintillator.
}
\fi
