\documentclass{article}
%\documentclass[12pt]{article}
\usepackage[utf8]{inputenc}
\usepackage{amsmath}
\usepackage{amssymb}
\usepackage{wasysym}
\usepackage{xcolor}
\usepackage{bm}
\usepackage{bbm}
\usepackage{gensymb}
\usepackage{graphicx}
\usepackage{caption}
\usepackage{subcaption}
\usepackage[superscript,nomove]{cite}
\usepackage{hyperref}
\usepackage{multirow}
\usepackage{enumitem}
\usepackage[margin=1in]{geometry}
\usepackage[linesnumbered,ruled,vlined]{algorithm2e}
\makeatletter 
\g@addto@macro{\@algocf@init}{\SetKwInOut{Parameter}{Arguments}} 
\makeatother 
\graphicspath{{figures/}}
\usepackage{subfiles} % Best loaded last in the preamble
%% Support for easy cross-referencing (e.g. \cref{sec:intro}
% configured with \AtEndPreamble as it needs to be called after hyperref
\usepackage[capitalize]{cleveref}
\crefname{section}{Sec.}{Secs.}
\Crefname{section}{Section}{Sections}
\Crefname{table}{Table}{Tables}
\crefname{table}{Tab.}{Tabs.}

\usepackage{float}

\usepackage{xr}
\externaldocument{main}

\begin{document}

\title{Kaleidoscopic Scintillation Event Imaging \\
\large Supplementary Material}

\date{}

\maketitle

\section{Derivations}
\subsection{Unweighted likelihood and gradient} \label{sec:like}

Denote an event location at $\bm{p_0}$.
In these derivations, we use camera coordinates (origin at lens). 
To transform from world coordinates (origin at pyramid apex) to camera 
coordinates, apply $z_c = z_{lens} - z_w$, where $z_{lens}$ is in world coordinates.
The event's mirror reflection $k$ is located at $\bm{p_k}=(x_k,y_k,z_k)=T_k\bm{p_0}$.
$\alpha_{xyk}$ denotes the coefficient in $T_k$ corresponding to $x_0$ and $y_k$.
$k=0$ refers to the event itself, and not a mirror reflection.
These derivations assume $\bm{\mu_k} = \left[-\frac{S_2}{z_k} x_k, -\frac{S_2}{z_k} y_k\right]$. 
%Modifications have to be made if using $\bm{\mu_k} = \left[\frac{S_2}{z_k} x_k, \frac{S_2}{z_k} y_k\right]$.



\begin{align}
x_k = \alpha_{xxk}x_0 + \alpha_{xyk}y_0 + \alpha_{xzk}z_{w0}
\end{align}
\begin{align}
y_k = \alpha_{yxk}x_0 + \alpha_{yyk}y_0 + \alpha_{yzk}z_{w0}
\end{align}
\begin{align}
z_k = z_\text{lens} - (\alpha_{zxk}x_0 + \alpha_{zyk}y_0 + \alpha_{zzk}z_{w0})
\end{align}


\begin{align}
\bm{\mu_k} = \left[-\frac{S_2}{z_k} x_k, -\frac{S_2}{z_k} y_k\right]
\end{align}
\begin{align}
\sigma_k = a \frac{AS_2}{S_1} \frac{|S_1-z_k|}{z_k}
\end{align}


\begin{align}
L(\bm{\theta};\bm{t},\bm{z}) &= \prod_{i=1}^N \prod_{k=0}^K \left[ \pi_k \mathcal{N}(\bm{t_i};\bm{\mu_k},\sigma_k^2) \right]^{\mathbbm{1}{(z_i=k)}} \\
&= \prod_{i=1}^N \prod_{k=0}^K \left[ \pi_k \frac{1}{2\pi{\sigma_k}^2} \text{exp}\left( -\frac{1}{2{\sigma_k}^2} (\bm{t_i}-\bm{\mu_k})^T(\bm{t_i}-\bm{\mu_k}) \right) \right]^{\mathbbm{1}{(z_i=k)}}
\end{align}

\begin{align}
(\bm{t_i}-\bm{\mu_k})^T(\bm{t_i}-\bm{\mu_k}) &= {t_{ix}}^2 - 2t_{ix}\left(-S_2\frac{x_k}{z_{k}}\right) + \left(-S_2\frac{x_k}{z_{k}}\right)^2 + {t_{iy}}^2 - 2t_{iy}\left(-S_2\frac{y_k}{z_{k}}\right) + \left(-S_2\frac{y_k}{z_{k}}\right)^2 \\
&= {t_{ix}}^2 + 2t_{ix}S_2\frac{x_k}{z_{k}} + {S_2}^2\left(\frac{x_k}{z_{k}}\right)^2 + {t_{iy}}^2 + 2t_{iy}S_2\frac{y_k}{z_{k}} + {S_2}^2\left(\frac{y_k}{z_{k}}\right)^2
\end{align}


\begin{align}
Q = \sum_i \sum_k r_{ik} \left[ \text{log}\pi_k - \text{log}(2\pi{\sigma_k}^2) - \frac{{\sigma_k}^{-2}}{2}(\bm{t_i}-\bm{\mu_k})^T(\bm{t_i}-\bm{\mu_k}) \right]
\end{align}

\begin{align}
\frac{x_k}{z_k} = \frac{\alpha_{xxk}x_0 + \alpha_{xyk}y_0 + \alpha_{xzk}z_0}{z_\text{lens} - (\alpha_{zxk}x_0 + \alpha_{zyk}y_0 + \alpha_{zzk}z_0)}
\end{align}

\begin{align}
\frac{y_k}{z_k} = \frac{\alpha_{yxk}x_0 + \alpha_{yyk}y_0 + \alpha_{yzk}z_0}{z_\text{lens} - (\alpha_{zxk}x_0 + \alpha_{zyk}y_0 + \alpha_{zzk}z_0)}
\end{align}



$u_x = x_k = \alpha_{xxk}x_0 + \alpha_{xyk}y_0 + \alpha_{xzk}z_0$

${u_x}'(x_0) = \alpha_{xxk}$

${u_x}'(y_0) = \alpha_{xyk}$

${u_x}'(z_0) = \alpha_{xzk}$

$u_y = y_k = \alpha_{yxk}x_0 + \alpha_{yyk}y_0 + \alpha_{yzk}z_0$

${u_y}'(x_0) = \alpha_{yxk}$

${u_y}'(y_0) = \alpha_{yyk}$

${u_y}'(z_0) = \alpha_{yzk}$

$v_z = z_k = z_\text{lens} - (\alpha_{zxk}x_0 + \alpha_{zyk}y_0 + \alpha_{zzk}z_0)$

${v_z}'(x_0) = -\alpha_{zxk}$

${v_z}'(y_0) = -\alpha_{zyk}$

${v_z}'(z_0) = -\alpha_{zzk}$

\begin{align}
\frac{\partial}{\partial x_0}\left(\frac{x_k}{z_k}\right) = \frac{{u_x}'v_z -u_x{v_z}'}{v^2}  = \frac{\alpha_{xxk} z_k + \alpha_{zxk} x_k}{{z_k}^2}
\end{align}

\begin{align}
\frac{\partial}{\partial y_0}\left(\frac{x_k}{z_k}\right) = \frac{\alpha_{xyk} z_k + \alpha_{zyk} x_k}{{z_k}^2}
\end{align}

\begin{align}
\frac{\partial}{\partial z_0}\left(\frac{x_k}{z_k}\right) = \frac{\alpha_{xzk} z_k + \alpha_{zzk} x_k}{{z_k}^2}
\end{align}




\begin{align}
\frac{\partial}{\partial x_0}\left(\frac{y_k}{z_k}\right) = \frac{{u_y}'v_z -u_y{v_z}'}{v^2}  = \frac{\alpha_{yxk} z_k + \alpha_{zxk} y_k}{{z_k}^2}
\end{align}

\begin{align}
\frac{\partial}{\partial y_0}\left(\frac{y_k}{z_k}\right) = \frac{\alpha_{yyk} z_k + \alpha_{zyk} y_k}{{z_k}^2}
\end{align}

\begin{align}
\frac{\partial}{\partial z_0}\left(\frac{y_k}{z_k}\right) = \frac{\alpha_{yzk} z_k + \alpha_{zzk} y_k}{{z_k}^2}
\end{align}




\begin{align}
\frac{\partial}{\partial x_0}\left(\left(\frac{x_k}{z_k}\right)^2\right) = 2 \frac{x_k}{z_k} \frac{\partial}{\partial x_0}\left(\frac{x_k}{z_k}\right)
\end{align}

\begin{align}
\frac{\partial}{\partial y_0}\left(\left(\frac{x_k}{z_k}\right)^2\right) = 2 \frac{x_k}{z_k} \frac{\partial}{\partial y_0}\left(\frac{x_k}{z_k}\right)
\end{align}

\begin{align}
\frac{\partial}{\partial z_0}\left(\left(\frac{x_k}{z_k}\right)^2\right) = 2 \frac{x_k}{z_k} \frac{\partial}{\partial z_0}\left(\frac{x_k}{z_k}\right)
\end{align}




\begin{align}
\frac{\partial}{\partial x_0}\left(\left(\frac{y_k}{z_k}\right)^2\right) = 2 \frac{y_k}{z_k} \frac{\partial}{\partial x_0}\left(\frac{y_k}{z_k}\right)
\end{align}

\begin{align}
\frac{\partial}{\partial y_0}\left(\left(\frac{y_k}{z_k}\right)^2\right) = 2 \frac{y_k}{z_k} \frac{\partial}{\partial y_0}\left(\frac{y_k}{z_k}\right)
\end{align}

\begin{align}
\frac{\partial}{\partial z_0}\left(\left(\frac{y_k}{z_k}\right)^2\right) = 2 \frac{y_k}{z_k} \frac{\partial}{\partial z_0}\left(\frac{y_k}{z_k}\right)
\end{align}





\begin{align}
\sigma_k = a \frac{AS_2}{S_1} \frac{|S_1-z_k|}{z_k}
\end{align}

$ \frac{|S_1-z_k|}{z_k} = \frac{|S_1 - z_\text{lens} + \alpha_{zxk}x_0 + \alpha_{zyk}y_0 + \alpha_{zzk}z_0|}{z_\text{lens} - (\alpha_{zxk}x_0 + \alpha_{zyk}y_0 + \alpha_{zzk}z_0)} $

$ u = S_1 - z_\text{lens} + \alpha_{zxk}x_0 + \alpha_{zyk}y_0 + \alpha_{zzk}z_0 $


\begin{align}
\frac{\partial \sigma_k}{\partial x_0} = a\frac{AS_2}{S_1} \frac{\partial}{\partial x_0} \frac{|S_1-z_k|}{z_k} =  a\frac{AS_2}{S_1} \frac{\alpha_{zxk}sign(u)z_k - |u|(-\alpha_{zxk})}{{z_k}^2} =  a\frac{AS_2}{S_1} \frac{\alpha_{zxk}sign(u)z_k + \alpha_{zxk}|u|}{{z_k}^2}
\end{align}

\begin{align}
\frac{\partial \sigma_k}{\partial y_0} = a\frac{AS_2}{S_1} \frac{\partial}{\partial y_0} \frac{|S_1-z_k|}{z_k} =  a\frac{AS_2}{S_1} \frac{\alpha_{zyk}sign(u)z_k + \alpha_{zyk}|u|}{{z_k}^2}
\end{align}

\begin{align}
\frac{\partial \sigma_k}{\partial z_0} =  a\frac{AS_2}{S_1} \frac{\partial}{\partial z_0} \frac{|S_1-z_k|}{z_k} =   a\frac{AS_2}{S_1} \frac{\alpha_{zzk}sign(u)z_k + \alpha_{zzk}|u|}{{z_k}^2}
\end{align}


\begin{align} \label{eqn:Q_no_weight}
Q &= \sum_i \sum_k r_{ik} \left[ \text{log}\pi_k - \text{log}(2\pi{\sigma_k}^2) - \frac{{\sigma_k}^{-2}}{2}(\bm{t_i}-\bm{\mu_k})^T(\bm{t_i}-\bm{\mu_k}) \right] \\
&= \sum_i \sum_k r_{ik} \left[ \text{log}\pi_k - \text{log}(2\pi{\sigma_k}^2) - \frac{{\sigma_k}^{-2}}{2} \left({t_{ix}}^2 + 2t_{ix}S_2\frac{x_k}{z_k} + {S_2}^2\left(\frac{x_k}{z_k}\right)^2 + {t_{iy}}^2 + 2t_{iy}S_2\frac{y_k}{z_k} + {S_2}^2\left(\frac{y_k}{z_k}\right)^2 \right) \right]
\end{align}

\begin{equation}
f(x_k,y_k,z_k) = \left({t_{ix}}^2 + 2t_{ix}S_2\frac{x_k}{z_k} + {S_2}^2\left(\frac{x_k}{z_k}\right)^2 + {t_{iy}}^2 + 2t_{iy}S_2\frac{y_k}{z_k} + {S_2}^2\left(\frac{y_k}{z_k}\right)^2 \right)
\end{equation}

\begin{equation}
\frac{\partial f}{\partial x_0} = 2t_{ix}S_2 \frac{\partial}{\partial x_0} \left(\frac{x_k}{z_k}\right) + {S_2}^2 \frac{\partial}{\partial x_0}\left(\left(\frac{x_k}{z_k}\right)^2\right) + 2t_{iy}S_2 \frac{\partial}{\partial x_0} \left(\frac{y_k}{z_k}\right) + {S_2}^2 \frac{\partial}{\partial x_0}\left(\left(\frac{y_k}{z_k}\right)^2\right)
\end{equation}

\begin{equation}
\frac{\partial f}{\partial y_0} = 2t_{ix}S_2 \frac{\partial}{\partial y_0} \left(\frac{x_k}{z_k}\right) + {S_2}^2 \frac{\partial}{\partial y_0}\left(\left(\frac{x_k}{z_k}\right)^2\right) + 2t_{iy}S_2 \frac{\partial}{\partial y_0} \left(\frac{y_k}{z_k}\right) + {S_2}^2 \frac{\partial}{\partial y_0}\left(\left(\frac{y_k}{z_k}\right)^2\right)
\end{equation}

\begin{equation}
\frac{\partial f}{\partial z_0} = 2t_{ix}S_2 \frac{\partial}{\partial z_0} \left(\frac{x_k}{z_k}\right) + {S_2}^2 \frac{\partial}{\partial z_0}\left(\left(\frac{x_k}{z_k}\right)^2\right) + 2t_{iy}S_2 \frac{\partial}{\partial z_0} \left(\frac{y_k}{z_k}\right) + {S_2}^2 \frac{\partial}{\partial z_0}\left(\left(\frac{y_k}{z_k}\right)^2\right)
\end{equation}

\begin{align} \label{eqn:dx_no_weight}
\frac{\partial Q}{\partial x_0} = \sum_i \sum_k r_{ik} \left[ -2{\sigma_k}^{-1}\frac{\partial \sigma_k}{\partial x_0} - \left( -{\sigma_k}^{-3} \frac{\partial \sigma_k}{\partial x_0}f + \frac{{\sigma_k}^{-2}}{2} \frac{\partial f}{\partial x_0}  \right) \right] 
\end{align}

\begin{align} \label{eqn:dy_no_weight}
\frac{\partial Q}{\partial y_0} = \sum_i \sum_k r_{ik} \left[ -2{\sigma_k}^{-1}\frac{\partial \sigma_k}{\partial y_0} - \left( -{\sigma_k}^{-3} \frac{\partial \sigma_k}{\partial y_0}f + \frac{{\sigma_k}^{-2}}{2} \frac{\partial f}{\partial y_0}  \right) \right] 
\end{align}

\begin{align} \label{eqn:dz_no_weight}
\frac{\partial Q}{\partial z_0} = \sum_i \sum_k r_{ik} \left[ -2{\sigma_k}^{-1}\frac{\partial \sigma_k}{\partial z_0} - \left( -{\sigma_k}^{-3} \frac{\partial \sigma_k}{\partial z_0}f + \frac{{\sigma_k}^{-2}}{2} \frac{\partial f}{\partial z_0}  \right) \right] 
\end{align}



\subsection{Weighted likelihood} \label{sec:weighted_like}

$w_i$ is the weight assigned to photon sample $i$.
$w_i = \sum\limits_{j \in S_i^q} \text{exp} \left( -\frac{||\bm{t_i}-\bm{t_j}||_2^2}{\nu} \right)$
where $S_i^q$ is the set of $q$ nearest neighbors of photon $i$, and $\nu$ is a 
positive scalar.



\begin{align}
L(\bm{\theta};\bm{t},\bm{z}) &= \prod_{i=1}^N \prod_{k=0}^K \left[ \pi_k \mathcal{N}(\bm{t_i};\bm{\mu_k},\frac{1}{w_i}\sigma_k^2) \right]^{\mathbbm{1}{(z_i=k)}} \\
&= \prod_{i=1}^N \prod_{k=0}^K \left[ \pi_k \frac{w_i}{2\pi{\sigma_k}^2} \text{exp}\left( -\frac{w_i}{2{\sigma_k}^2} (\bm{t_i}-\bm{\mu_k})^T(\bm{t_i}-\bm{\mu_k}) \right) \right]^{\mathbbm{1}{(z_i=k)}}
\end{align}

\begin{align}
Q = \sum_i \sum_k r_{ik} \left[ \text{log}(\pi_k) + \text{log}(w_i) - \text{log}(2\pi{\sigma_k}^2) - \frac{w_i{\sigma_k}^{-2}}{2}(\bm{t_i}-\bm{\mu_k})^T(\bm{t_i}-\bm{\mu_k}) \right]
\end{align}

\begin{align}
\frac{\partial Q}{\partial x_0} = \sum_i \sum_k r_{ik} \left[ -2{\sigma_k}^{-1}\frac{\partial \sigma_k}{\partial x_0} - w_i\left( -{\sigma_k}^{-3} \frac{\partial \sigma_k}{\partial x_0}f + \frac{{\sigma_k}^{-2}}{2} \frac{\partial f}{\partial x_0}  \right) \right] 
\end{align}

\begin{align}
\frac{\partial Q}{\partial y_0} = \sum_i \sum_k r_{ik} \left[ -2{\sigma_k}^{-1}\frac{\partial \sigma_k}{\partial y_0} - w_i\left( -{\sigma_k}^{-3} \frac{\partial \sigma_k}{\partial y_0}f + \frac{{\sigma_k}^{-2}}{2} \frac{\partial f}{\partial y_0}  \right) \right] 
\end{align}

\begin{align}
\frac{\partial Q}{\partial z_0} = \sum_i \sum_k r_{ik} \left[ -2{\sigma_k}^{-1}\frac{\partial \sigma_k}{\partial z_0} - w_i\left( -{\sigma_k}^{-3} \frac{\partial \sigma_k}{\partial z_0}f + \frac{{\sigma_k}^{-2}}{2} \frac{\partial f}{\partial z_0}  \right) \right] 
\end{align}


\section{Experimental parameter values} \label{sec:params}

\textbf{Algorithm parameter values and stopping criteria.}
We use $q=10$ nearest neighbors and $\nu=10$ pixels for assigning photon weights.
Up to ten possible event locations are evaluated in the initialization procedure 
equispaced over the scintillator's depth.
During one M-step, we run gradient ascent for 1,000 steps with a step size of 1e-7.
We run the EM algorithm until the distance in the estimated event location between 
consecutive steps is less than 0.01 mm, or until 100 steps are taken.

\noindent
\textbf{Experimental camera parameter calibration.}
Experimental camera parameter values for $S_1$, $S_2$, and $a$ are calibrated by 
optimizing $Q$ in a grid search on a selected experimental image.
We select the image shown in Supplementary \cref*{fig:calibration}, where the 
event is approximately centered with evident mirror reflections, and we manually 
remove dark counts.
The grid search is performed over event locations at $x$-$y$ coordinates 
$(0,0)$ mm and $z$-coordinates that span the depth of the scintillator.
This resulted in $S_1=37.24$ mm, $S_2=109.48$ mm, and $a=0.056$ at event location 
(0, 0, 2.61) mm.


\section{Supplementary Figures}

\begin{figure}
\centering
\includegraphics[width=.5\linewidth]{trunc_teaser.jpg}
\caption{\textbf{Image truncation example in 3D.}
The image of an event and one mirror reflection is shown. 
Light corresponding to the mirror reflection is truncated at the mirror's edge, 
resulting in a truncation line in the image.
The truncation line only applies to that mirror reflection.}
\label{fig:trunc_teaser}
\end{figure}

\begin{figure}
\centering
\includegraphics[width=.5\linewidth]{dark_counts_hist_median=4.png}
\caption{\textbf{Histogram of dark counts per experimental image.} 
A median of 4 dark counts per image after zeroing hot pixels is observed out of 
130,000 images taken in the dark with no gamma-ray source present.} 
\label{fig:dark_counts_hist}
\end{figure}


\begin{figure}
\centering
\includegraphics[width=\linewidth]{capture_counts_hist.pdf}
\caption{\textbf{Histogram of counts per experimental image with the gamma-ray source present.} A median of 4 counts 
per image after zeroing hot pixels is observed out of 13,000,000 images taken with 
the gamma-ray source present. 
a) The full histogram. 
b) The histogram pertaining to at least 60 counts in an image.} 
\label{fig:cap_counts_hist}
\end{figure}


\begin{figure*}
\centering
\includegraphics[width=\linewidth]{calibration.pdf}
\caption{\textbf{Selected experimental calibration image.} a) The original image. b) The image after manually 
removing dark counts overlaid with the Gaussian components found during the 
calibration procedure.
Each dashed red circle is centered on the Gaussian component's mean. 
The inner and outer circles are one and two standard deviations in radius, respectively.
Pixels with a photon are enlarged with a $3 \times 3$ filter for visualization purposes.
} 
\label{fig:calibration}
\end{figure*}

\begin{figure*}
\centering
\includegraphics[width=\linewidth]{capture_examples_sup.png}
\caption{\textbf{Additional selected experimental images.} Experimental images overlaid with the algorithm's estimated Gaussian components.
Each dashed red circle is centered on the Gaussian component's mean. 
The inner and outer circles are one and two standard deviations in radius, respectively.
Pixels with a photon are enlarged with a $3 \times 3$ filter for visualization purposes.
} 
\label{fig:example_figures_sup}
\end{figure*}


\begin{figure}
\centering
\includegraphics[width=\linewidth]{crossval_error_v2.pdf}
\caption{\textbf{Agreement in event location measurements.} Histogram of distances 
between mean estimated event location and each image's estimated event location 
after mirror reflection removals. Median, mean, stdev: 0.12 mm, 0.24 mm, 0.34 mm. 24,761 distances.
} 
\label{fig:crossval_error}
\end{figure}


\begin{figure*}
\centering
\includegraphics[width=\linewidth]{fixed_init_results.pdf}
\caption{\textbf{Optimization convergence.}
(a) Distances in estimated event locations between the regular and fixed initialization methods. Median, mean, stdev: 0.007 mm 0.38 mm, 0.80 mm. 4,351 events.
(b) Number of steps taken in the EM algorithm from the regular initialization point. Median, mean, stdev: 4 steps, 9.3 steps, 14.2 steps. 4,351 events.
(c) Number of steps taken in the EM algorithm from the fixed initialization point. Median, mean, stdev: 8 steps, 13.5 steps, 14.9 steps. 4,351 events.
} 
\label{fig:convergence}
\end{figure*}


\begin{figure}
\centering
\includegraphics[width=.4\linewidth]{scintillator_pic.jpg}
\caption{\textbf{The kaleidoscopic scintillator used in experiments.}}
\label{fig:scintillator}
\end{figure}

\iffalse
\begin{figure}
\centering
\includegraphics[width=.7\linewidth]{setup_pic.jpg}
\caption{The experimental setup without the gamma-ray source.} 
\label{fig:setup}
\end{figure}
\fi

\begin{figure}
\centering
\includegraphics[width=.5\linewidth]{20250613_spad512_focus.png}
\caption{\textbf{Experimental focus.} A view of how the camera is focused on the scintillator in the experiments. The edges between mirrors are visible.} 
\label{fig:experiment_focus}
\end{figure}


\end{document}

