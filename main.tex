\documentclass{article}
%\documentclass[12pt]{article}
\usepackage[utf8]{inputenc}
\usepackage{amsmath}
\DeclareMathOperator*{\argmax}{\arg\!\max}
\DeclareMathOperator*{\argmin}{\arg\!\min}
\usepackage{amssymb}
\usepackage{bbm}
\usepackage{wasysym}
\usepackage{xcolor}
\usepackage{gensymb}
\usepackage{graphicx}
\usepackage{bm}
\usepackage{caption}
\usepackage{subcaption}
%\usepackage{algorithm}
\usepackage{algorithm2e}
\usepackage{algpseudocode}
\usepackage{hyperref}
\usepackage{multirow}
\usepackage[superscript,nomove]{cite}
\usepackage{enumitem}
\usepackage[normalem]{ulem}
\usepackage{ifthen}
\newboolean{figs_in_text}
\setboolean{figs_in_text}{false}
\usepackage[margin=1in]{geometry}
%\usepackage[linesnumbered,ruled,vlined]{algorithm2e}
\usepackage[makeroom]{cancel}
\makeatletter 
\g@addto@macro{\@algocf@init}{\SetKwInOut{Parameter}{Arguments}} 
\makeatother 
\graphicspath{{figures/}}
%% Support for easy cross-referencing (e.g. \cref{sec:intro}
% configured with \AtEndPreamble as it needs to be called after hyperref
\usepackage[capitalize]{cleveref}
\crefname{section}{Sec.}{Secs.}
\Crefname{section}{Section}{Sections}
\Crefname{table}{Table}{Tables}
\crefname{table}{Tab.}{Tabs.}

\usepackage{subfiles} % Best loaded last in the preamble

\usepackage{xr}
\externaldocument{supplement}

\usepackage{authblk}

% line numbers
\usepackage{lineno}
\linenumbers

\begin{document}

\title{Kaleidoscopic Scintillation Event Imaging}

\author[1]{Alex Bocchieri*} % (alexb826@gmail.com)}
\author[2,3]{John Mamish}
\author[2]{David Appleyard}
\author[4,5]{Andreas Velten*}% (velten@wisc.edu)}

\affil[1]{Department of Computer Sciences, University of Wisconsin - Madison, Madison, WI, USA}
\affil[2]{Ubicept, Sun Prairie, WI, USA}
\affil[3]{College of Computing, Georgia Institute of Technology, Atlanta, GA, USA}
\affil[4]{Department of Biostatistics and Medical Informatics, University of Wisconsin - Madison, Madison, WI, USA}
\affil[5]{Department of Electrical and Computer Engineering, University of Wisconsin - Madison, Madison, WI, USA}


\date{}

\maketitle
\begin{abstract}
A scintillator emits visible light fluorescence when ionizing radiation deposits 
energy during a scintillation event.
Single-photon camera designs for measuring scintillation events are limited by low 
levels of light collection.
We propose a kaleidoscopic scintillator geometry with specular surfaces to 
increase light collection in a single-photon camera while preserving the event's 
spatial information.
The kaleidoscopic geometry creates mirror reflections of the event in known 
locations for a given event location.
We introduce theory for imaging a kaleidoscopic scintillation event, 
a probabilistic model for the image of the event and its mirror reflections,
and an algorithm to estimate the event's location.
We confirm the presence of mirror reflections in accordance with the 
theory and validate the algorithm on experimental data.
The kaleidoscopic scintillator design may provide sufficient light collection to 
perform advanced radiation imaging techniques using a single-photon camera.
\end{abstract}
\section{Introduction}

High energy particles are created in nuclear processes including man-made 
devices, cosmic radiation, and natural interactions. 
There is a wide variety of particles with varying properties. 
Detecting and characterizing them is crucial for a broad range of applications 
including nuclear security \cite{vetter2018gamma}, nuclear reactor and stockpile imaging \cite{beaumont2015high}, 
medical imaging \cite{gonzalez2021evolution}, 
archeology \cite{menichelli2007scintillating,ryzewski2013neutron}, 
and astronomy \cite{schonfelder1984imaging}. 
A particularly useful property of these particles is that many of them can 
penetrate through dense materials and therefore allow us to image through 
barriers and inside solid structures. 
The same penetrating properties, however, make it challenging to build a camera 
that can perform imaging or vision using high energy particles. 
A detector needs to have a large volume and density for a particle to interact. 
At the same time, we would like to know the location and shape of the interaction if it happens. 
One common approach to achieve this is by using a scintillator.

A scintillator is a radiation detector that converts ionizing radiation into 
visible light.
It provides mass for an incident particle to be absorbed and detected with a photosensor.
While a particle is propagating through a scintillator, it may deposit energy and 
cause a ``scintillation event". 
In the case of gamma-ray radiation, a gamma-ray collides with an electron in the 
scintillator and causes the electron to recoil over a random walk.
Scintillation photons are emitted isotropically from the electron's path over a 
decay time.
The photons propagate out of the scintillator and are captured by a sensor.
Measuring the event, such as its position, time, and energy deposition, is 
performed with the obtained signal.
These measurements are then used in various downstream tasks to characterize 
the radiation source.

Scintillation light collection methods consider both scintillator hardware design  
and sensor characteristics to optimize event measurements for a given task.
In this work, we focus on the design of measuring events in a monolithic 
scintillator with a camera.
A drawback of this design is its low light collection compared to that from 
coupling a sensor directly to the scintillator's surface.
The reasons are that the event must be far enough from the lens to be imaged, and 
the lens adds a restrictive aperture and reduced range of acceptance angles for 
incident light.
This drawback has a smaller impact in tasks where measurements of individual 
particles or gamma-rays are not needed.
For example, in certain radiography tasks, the overall glow from ionizing 
radiation being absorbed in a scintillator after being attenuated in an object is 
used to form an image of the object.
In this case, increasing total exposure time can increase the signal to 
noise ratio (SNR), and low-speed cameras (CMOS, CCD, etc.) are adequate \cite{pleinert1997design,baker2014scintillator,adams2017gamma,balasubramanian2022x,gustschin2024event}.
On the other hand, measuring individual events requires high-speed, single-photon 
sensors due high particle incidence rates and the finite number of photons emitted 
by an event.
In this case, increasing exposure time does not increase light collection per 
event and decreases SNR.
Therefore, maximizing light throughput is crucial for this kind of task.
The ``Compton camera" \cite{kataoka2013handy, hosokoshi2019development, llosa2019sipm, parajuli2022development, kim2024comprehensive} 
and neutron scatter camera \cite{mascarenhas2006development,marleau2007advances,mascarenhas2009results,weinfurther2018model} 
are examples of devices that measure events from individual gamma-rays or neutrons.
These devices measure double-interaction events and use the kinematics of 
gamma-ray or neutron scattering to perform backprojection for localizing the 
radiation source. % \cite{haefner2015filtered, xu2006filtered, wilderman1998list}.
Recently, single-photon-avalanche-diode (SPAD) cameras have been of interest for 
measuring events for their high spatial and temporal resolution at the sensor \cite{bocchieri2024scintillation}.
SPAD cameras could potentially make high-resolution measurements of individual 
events in a thick scintillator volume and perform advanced radiation imaging 
techniques, but they are limited by low levels of light collection.

Reflective surfaces and scintillator geometry have been considered for increasing 
light collection in various designs.
%Retroreflectors have been shown to improve light collection in parallelpiped and 
%pyramidal frustum scintillators and a position-sensitive photomultiplier (PSPMT) 
%sensor \cite{ros2014retroreflector}. 
Retroreflectors have been shown to improve light collection along with spatial 
and energy resolution in monolithic scintillators.  \cite{mcelroy2002use,heemskerk2009micro,ros2014retroreflector,gonzalez2017performance}.
Specular and diffuse reflectors have been tested in the context of improving 
reflection models for simulation purposes \cite{janecek2010simulating,roncali2017integrated,trigila2021optimization}.
These studies use designs where a sensor is coupled to the scintillator's surface.
To the best of our knowledge, reflective surfaces have not been reported for 
collecting additional light in camera designs.
Adding diffuse surfaces will reflect light into the camera but obfuscate the 
event's image and cause a loss in spatial information.
Retroreflector surfaces better preserve spatial information but can no more than 
double the amount of captured light.
Retroreflectors also add blur that may degrade spatial resolution.
Current camera designs only capture light emitted directly to the camera,
which is a small fraction of all the light emitted from an event 
\cite{bocchieri2024scintillation, yamamoto2023development, d2021novel, gao2023novel, losko2021new,adams2017gamma,balasubramanian2022x,gustschin2024event}.
The goal of this work is to increase light collection by capturing additional 
light from an event that is emitted away from the camera.
The challenge lies in doing so without losing the event's spatial information.
%Another option for increasing light collection would be to surround the 
%scintillator with multiple cameras.
%However, the extra camera hardware increases overall cost and complexity and may 
%cause incident particles to scatter before reaching the scintillator.

We propose to use a kaleidoscopic scintillator geometry with specular surfaces to 
redirect light into the camera.
A kaleidoscope consists of planar mirrors oriented like a pyramid frustum.
%that produce multiple images of an object from different viewpoints.
Light from inside the kaleidoscope reflects one or more times before 
propagating out of the pyramid's open base and into a camera.
The resulting image contains direct light from the event and indirect light from 
the event's mirror reflections. 
Mirror reflections appear as events in locations determined by the kaleidoscope's geometry.
Mirrors redirect light and preserve its structure, thus preserving the event's 
spatial information.
The kaleidoscope can increase light collection by multiple factors.
In this work, we demonstrate an improvement in light collection by up to about 
five times that compared to imaging direct light only.
The task is to determine the event's 3D position from the resulting image.

%With sufficient light, complex signatures such as multi-interaction events, 
%particle tracks, and Cherenkov photons could be measured.


Mirrors and kaleidoscopic designs have been used for stereo vision \cite{nene1998stereo, gluckman1999planar, gluckman2002rectified} 
and multi-view 3D reconstructions \cite{reshetouski2011three, ahn2021kaleidoscopic, ahn2023neural, takahashi2021structure, bangay2004kaleidoscope, mitsumoto19923}
of an extended object using one picture from a single camera.
A ``light-trap" design consists of mirrors oriented such that light entering the 
trap reaches nearly every position inside the trap \cite{xu2018trapping}.
Time-of-flight is used to reconstruct a 3D object inside the trap.
In fact, the pyramid-shaped light-trap was found to provide the best object 
coverage, which is the same shape as the kaleidoscope we use in this work.
Here, we apply the kaleidoscope in the photon-starved setting of scintillation 
event imaging and form a radiation detection problem as a computer vision problem.


In this paper, we make the following contributions: 
\begin{itemize}
\item A new scintillator design for increasing light collection in a camera while 
preserving the event's spatial information.
\item Theory for modeling light from an event in a kaleidoscopic scintillator that 
arrives at the camera sensor.
\item An algorithm to estimate an event's location in a kaleidoscopic scintillator. 
The algorithm is validated on experimental data captured with a SPAD camera and a 
gamma-ray source.
\item \textcolor{red}{Is the captured dataset a contribution?}
\end{itemize}
An overview of the method is illustrated in \cref{fig:teaser}.


\ifthenelse{\boolean{figs_in_text}}{

\begin{figure}
\centering
\includegraphics[width=\linewidth]{teaser_pipeline.jpg}
\caption{\textbf{Overview of image capture and event localization method.}
An image is composed of light that is emitted from the scintillation event and 
reaches the camera either directly or after reflecting off mirrors of a 
kaleidoscopic scintillator.
The spatial relationship between the event and mirror reflections is embedded in a 
Gaussian mixture model whose likelihood is maximized to estimate the event's location.
} 
\label{fig:teaser}
\end{figure}

}{}


\section{Results}
\iffalse
We test a kaleidoscopic scintillator with a square pyramid geometry (four side-surfaces).
An image is assumed to contain the event and at most one mirror reflection per 
side-surface of the kaleidoscope.
Each event and mirror reflection appears as a defocused point source of light in 
the image.
Light from mirror reflections may be truncated along lines on the sensor 
corresponding to mirrors' edges or completely removed, as described by the theory 
introduced in the ``Methods" section.
Our algorithm determines which mirror reflections are present in an image and 
estimates the event's location by running the EM algorithm on a Gaussian mixture 
model (GMM) parameterized in terms of the event's location.
\fi

Results are summarized below. See the ``Methods" section for more information on 
the theory of kaleidoscopic light collection, simulated image generation, 
algorithm, and experiments.

A simulated image overlaid with each mirror reflection's acceptance zone of where 
photons can arrive is shown in \cref{fig:trunc_examples}. 
This figure is used to validate the theory on kaleidoscopic light collection and 
image truncations. 

Selected experimental images overlaid with the algorithm's estimated Gaussian 
components are shown in \cref{fig:example_figures} to validate the kaleidoscopic 
model and the presence of mirror reflections.

The distances between mean and individual estimated event locations in groups of 
experimental images of removed mirror reflections are reported in \cref{fig:crossval_error}.
Statistics on the distances in the single- and double-removed mirror reflection 
group over different ranges of photon counts are reported in \cref{tab:crossval_thresh_error}.
Distances between estimated event locations using the regular and fixed 
initialization methods, as well as the number of EM steps taken, are shown in \cref{fig:convergence}.
These results are for validating that the algorithm is estimating the event's 
location and converging in optimization.


\ifthenelse{\boolean{figs_in_text}}{

\begin{figure}
\centering
\includegraphics[width=\linewidth]{trunc_examples.pdf}
\caption{\textbf{Simulated kaleidoscopic image with theoretical acceptance zones.} 
Acceptance zones are derived for each mirror reflection and overlaid on the image 
for the (a) +$x$, (b) +$y$, (c) -$x$, and (d) -$y$ mirror reflections.} 
\label{fig:trunc_examples}
\end{figure}



\begin{figure*}
\centering
\includegraphics[width=\linewidth]{example_figures.pdf}
\caption{\textbf{Selected experimental images.} Experimental images overlaid with the algorithm's estimated Gaussians.
Each dashed red circle is centered on the Gaussian component's mean. 
The inner and outer circles are one and two standard deviations in radius, respectively.
Pixels with a photon are enlarged with a $3 \times 3$ filter for visualization purposes.
} 
\label{fig:example_figures}
\end{figure*}



\begin{figure*}
\centering
\includegraphics[width=\linewidth]{crossval_error.pdf}
\caption{\textbf{Agreement in event location measurements.} Histogram of distances between mean estimated event location and each image's estimated event location after mirror reflection removals. 
(a) Distances including both single and double mirror reflection removals. Mean $\pm$ stdev: 0.24 $\pm$ 0.34 mm, 24,761 distances.
(b) Distances including single mirror reflection removal. Mean $\pm$ stdev: 0.16 $\pm$ 0.28 mm, 11,255 distances.
(c) Distances including double mirror reflection removal. Mean $\pm$ stdev: 0.27 $\pm$ 0.35 mm, 15,757 distances.
} 
\label{fig:crossval_error}
\end{figure*}

\begin{figure*}
\centering
\includegraphics[width=\linewidth]{fixed_init_results.pdf}
\caption{\textbf{Optimization convergence.}
(a) Distances in estimated event locations between the regular and fixed initialization methods. Mean $\pm$ stdev: $0.38 \pm 0.80$ mm, 4,351 events.
(b) Nummber of steps taken in the EM algorithm from the regular initialization point. Mean $\pm$ stdev: $9.3 \pm 14.2$ steps, 4,351 events.
(c) Nummber of steps taken in the EM algorithm from the fixed initialization point. Mean $\pm$ stdev: $13.5 \pm 14.9$ steps, 4,351 events.
} 
\label{fig:convergence}
\end{figure*}


\begin{table}[h!]
\centering
\begin{tabular}{|c|ccccc|}
\hline
Counts in a test image & 60 - 79 & 80 - 99 & 100 - 119 & 120 - 139 & 140 - 159 \\
\hline
Mean (mm)                & 0.28 & 0.26 & 0.19 & 0.19 & 0.15 \\
Standard dev.  (mm)      & 0.36 & 0.35 & 0.30 & 0.29 & 0.21  \\
Number of distances    & 9,317  & 7,524  & 4,774  & 2,354  & 792 \\
\hline
\end{tabular}
\caption{\textbf{Agreement in event location measurements.} Mean and standard deviation of distances between mean estimated 
event location and each image's estimated event location after both single and 
double mirror reflection removals reported over different ranges of counts in a 
test image.}
\label{tab:crossval_thresh_error}
\end{table}

}{}



%\section{Related work}
\textbf{Kaleidoscopic methods:}
Mirrors and kaleidoscopic designs have been used for stereo vision \cite{nene1998stereo, gluckman1999planar, gluckman2002rectified} 
and multi-view 3D reconstructions \cite{reshetouski2011three, ahn2021kaleidoscopic, ahn2023neural, takahashi2021structure, bangay2004kaleidoscope, mitsumoto19923}
of an extended object using one picture from a single camera.
A ``light-trap" design consists of mirrors oriented such that light entering the trap reaches nearly every position inside the trap \cite{xu2018trapping}.
Time-of-flight is used to reconstruct a 3D object inside the trap.
In fact, the pyramid-shaped light-trap was found to provide the best object coverage, which is the same shape as the kaleidoscope we use in this work.
Here, we apply the kaleidoscope in the photon-starved setting of scintillation 
event imaging and form a radiation detection problem as a computer vision problem.

\noindent
\textbf{Scintillation light collection:}
Many scintillator and sensor configurations exist to optimize for certain 
measurement characteristics.
The scintillator volume can be segmented with inserted waveguides to constrain the emission of light onto a sensor coupled at the scintillator's surface \cite{kato2013novel, kishimoto2013development}.
Sensors can be placed on one or more surfaces of the scintillator \cite{yamaya2011sipm, kishimoto2013development}.
Reflective surfaces can be coupled to the scintillator to reflect light to the sensor \cite{gonzalez2017performance, folsom2021characterization, janecek2010simulating}.
In this work, we propose a new scintillator geometry with specular surfaces to 
increase light collection for a camera sensor.


\noindent
\textbf{Scintillator applications:}
A ``Compton camera" device \cite{kataoka2013handy, hosokoshi2019development, llosa2019sipm, parajuli2022development, kim2024comprehensive} 
measures scattering and absorption events of individual gamma-rays in a scintillator 
to perform Compton backprojection \cite{haefner2015filtered, xu2006filtered, wilderman1998list} for gamma-ray imaging.
The Compton camera and related detectors are used in applications for 
national security and nuclear safety \cite{vetter2018gamma}, as well as astronomy \cite{schonfelder1984imaging}.
Scintillators are also used in positron emission tomography (PET) to detect the 
coincidence of annihilation photons (511 keV gamma-rays) for reconstructing the 
medical image \cite{gonzalez2021evolution}.


%
\section{Kaleidoscopic event imaging theory} \label{sec:theory}

Without loss of generality, we consider a square pyramid scintillator throughout 
this paper. 
Each face of the scintillator, except for the base of the pyramid, is a specular surface.
We assume the scintillator's index of refraction is $n=1$. 
We correct for the scintillator's true $n>1$ by modeling a pyramid with decreased 
height and increased opening angle at the apex based on $n$.
The true depth of an event measured in the smaller apparent volume can then be 
corrected for using $n$.

\subsection{Imaging configuration and model}

The world coordinate system's origin is set at the pyramid's apex with the 
$z$-axis directed perpendicular toward the pyramid's base surface.
The camera coordinate system's origin is at the center of the lens with its 
$z$-axis directed toward the pyramid in the opposite direction of the world's $z$-axis.
The $x$ and $y$ axes among the two coordinate systems are in the same directions, 
so transforming a point between the world and camera coordinate systems is 
carried out by adding or subtracting the $z$-coordinate with the distance between 
the lens and the scintillator's apex.
We denote a $z$-coordinate in the world coordinate system as $z_w$ and camera 
coordinate system as $z_c$.
We use a thin lens to model the camera.
Three important planes to note are the focal plane, the thin lens plane, and the 
sensor plane.
The focal plane is set at the pyramid's apex at $z_w=0$ mm.
A thin lens with diameter $A$ is placed at a distance $S_1$ from the focal plane. 
The sensor is placed at a distance $S_2$ from the lens.
The lens' focal length is set to $f=(S_1^{-1}+S_2^{-1})^{-1}$.
These parameters are illustrated in \cref{fig:optical_config}.

Throughout this paper, the scintillator is oriented so that its base surface's 
edges are parallel with the sensor's respective edges. 
In this manner, the normal vector for each specular surface of the scintillator is 
aligned along the $x$ or $y$ axes.
We denote the scintillator's four specular surfaces as +$x$, +$y$, -$x$, or -$y$ mirrors.

\ifthenelse{\boolean{figs_in_text}}{

\begin{figure}
\centering
\includegraphics[width=\linewidth]{optical_config_horiz.png}
\caption{\textbf{Imaging parameters and coordinate systems.} The figure only shows 
light emitted directly to the camera.} 
\label{fig:optical_config}
\end{figure}

}{}

A scintillation event is approximated as a point source of light.
Since the optical setup is constrained to short imaging distances, 
the images of an event and its mirror reflections exhibit defocus blur and 
have nonzero diameters.
An image's diameter on the sensor varies according to the event's distance from 
the focal plane, following the circle of confusion model.
For an event at $(x_0,y_0,z_{c0})$, 
the circle of confusion model yields
\begin{equation} \label{eqn:circ_of_conf}
c=A\frac{S_2}{S_1}\frac{|S_1-z_{c0}|}{z_{c0}}
\end{equation}
where $c$ is the image diameter at the sensor.

Light is emitted in all directions from the event.
Photons may arrive at the camera directly from the event or indirectly after 
reflecting off mirrors.
Direct photons form a complete image on the sensor, while indirect photons may 
form an image that is partially truncated or completely missing as described below.
Below, we describe the spatial relationship between an event and its mirror 
reflections, as well as their image on the sensor using a thin lens model.

\subsection{Mirror reflections and apertures}

Consider an event located at $\bm{p_0}$.
Mirror $k$ with normal vector $\bm{n_k}$ produces a mirror reflection located at 
$\bm{p_k}=\bm{T_k}\bm{p_0}$
where
\begin{equation} \label{eqn:ref_trans}
\bm{T_k}=\bm{I}_{3\times3} - 2\bm{n_k}\bm{n_k}^T
\end{equation}
is the mirror's transformation.
The mirror reflection of an event is also a point source of light.
The captured image of the mirror reflection is obtained from the photons that 
reflect off the mirror and into the camera, exhibiting the same defocus blur as if 
an event were located at $\bm{p_k}$.
However, due to the finite mirror size, the image on the sensor may be truncated 
along lines corresponding to the mirror's edges.
This occurs when $\bm{p_k}$ is behind another mirror from the camera's perspective.
The photons that are truncated from the image are those that reflect off the 
mirror adjacent to mirror $k$ near the shared edge.
Also, light in higher-order reflections over multiple mirrors may be stopped 
in previous reflections and also cause image truncations.
We term a mirror reflection's ``order" as the number of reflections its light 
underwent before forming.
Essentially, mirror $k$ behaves like an aperture to a light source at $\bm{p_k}$. 


Light from $\bm{p_k}$ that does not pass through 
mirror $k$ does not reach the camera and is truncated from the image.
\cref{fig:trunc_theory}a and c show a 2D view of the propagation of light over a first-order reflection without truncations.
\cref{fig:trunc_theory}b and d show how truncations form.
A 3D visualization of light truncation is shown in Supplementary \cref*{fig:trunc_teaser}.


\ifthenelse{\boolean{figs_in_text}}{

\begin{figure*}
\centering
\includegraphics[width=\linewidth]{truncation_theory.png}
\caption{\textbf{Mirror apertures and image truncations.} 
An event emits light onto a mirror that reflects into the camera. 
Some light might not reach the sensor due to finite mirrors and defocus blur.
Example cases include the following:
The mirror reflection is located within the focal plane (a,b,e) or beyond the focal plane (c,d).
All light that forms the mirror reflection reaches the sensor (a,c).
Some light that would form the mirror reflection is stopped at the mirror's edge 
and truncated on the sensor (b,d).
Some light that would form a double mirror reflection is stopped at both mirrors' 
edges and truncated on the sensor (e). 
The mirror for the second reflection is illustrated along the light's path.} 
\label{fig:trunc_theory}
\end{figure*}

}{}


In higher-order reflections, light reflecting off mirror $k$ can 
reflect off another mirror $l$ and generate mirror reflection 
at $\bm{p_l}=\bm{T_l} \bm{p_k}$.
The light incident on mirror $l$ from $\bm{p_k}$ passes through the aperture 
of mirror $k$.
If this light spans a partial area $A_l$ of mirror $l$, then mirror $l$'s aperture is $A_l$.
Otherwise, mirror $l$'s aperture is simply the mirror.
Mirror $l$'s aperture affects the light emitted from $\bm{p_l}$ toward another 
mirror for a higher-order reflection and toward the camera for imaging.
Higher-order reflections and imaging continue in the same manner.
A 2D view of truncations from multiple reflections is shown in \cref{fig:trunc_theory}e.


\subsection{Image truncation derivation} \label{sec:image_truncations}

Consider an event's mirror reflection located at $\bm{p_k}$ generated from mirror $k$.
Each edge of the mirror may impose a truncation line on the camera sensor.
A truncation line on the sensor, $l_\text{sensor}$, is determined as follows.
Denote a truncation plane, $P_{\text{trunc}}$, that contains $\bm{p_k}$ and the 
mirror's edge.
Denote the side of $P_{\text{trunc}}$ that faces away from the mirror using the 
normal vector $\bm{n_{\text{trunc}}}$.
Compute the intersection between $P_{\text{trunc}}$ and the focal plane to be the
truncation line at the focal plane, $l_\text{focal}$.
Project $\bm{n_{\text{trunc}}}$ onto the focal plane, denoted as 
$\bm{n_{\text{focal}}}$.
Scale $l_\text{focal}$ by the magnification, $m=-\frac{S_2}{S_1}$, to obtain $l_\text{sensor}$.
Scale $\bm{n_{\text{focal}}}$ by $m$ to obtain $\bm{n_{\text{sensor}}}$.

The truncation side of $l_\text{sensor}$ where photons do not arrive depends on 
which side of the focal plane that $\bm{p_k}$ is located.
If $\bm{p_k}$ is beyond the focal plane away from the lens at a distance $d>S_1$ 
from the lens, 
then the side of $l_\text{sensor}$ pointed to by $\bm{n_{\text{sensor}}}$ is 
truncated and contains no photon arrivals (\cref{fig:trunc_theory}b).
If $\bm{p_k}$ is between the focal plane and lens at a distance $d<S_1$ from the lens, 
then the side of $l_\text{sensor}$ opposite of $\bm{n_{\text{sensor}}}$ 
is truncated and contains no photon arrivals (\cref{fig:trunc_theory}d).

The area on the sensor where photons cannot arrive is the union of the
truncation sides of each truncation line.
We denote this area as the ``truncation zone" and its complement as the 
``acceptance zone" (\cref{fig:trunc_theory}b).
The event itself has no truncation zone on the sensor because its image is formed 
by light emitted directly to the camera without reflections.


\subsection{Image truncation theory validation}
We validate the theory on image truncations by observing that theoretically 
derived truncation lines align with photon arrivals in a simulated image.
We simulate the image of an event in a kaleidoscopic scintillator in the shape of 
a square pyramid using a thin lens and ray tracing. 
The kaleidoscope is 3.02 mm in height and has a 20 mm base length.
We set its index of refraction to 1 and place its apex at $z_w=0$ mm.
A thin lens with 30 mm focal length and 20 mm diameter is placed at $z_w=50$ mm.
A $512 \times 512$ sensor with 16 $\mu$m pixel pitch is placed at $z_w=125$ mm.
100,000 photons are emitted isotropically from $(0.25, 1, 2)$ mm (world 
coordinates).
We emit an unrealistically high number of photons so that image truncations are 
clearly observable.
Acceptance zones for each mirror reflection are derived and overlaid on the image.
The resulting image is shown in \cref{fig:trunc_examples}.

\begin{figure*}
\centering
\includegraphics[width=\linewidth]{trunc_horiz.png}
\caption{\textbf{Simulated kaleidoscopic image with theoretical acceptance zones.} 
Acceptance zones are derived for each mirror reflection and overlaid in gray on 
the image for the (a) +$x$, (b) +$y$, (c) -$x$, and (d) -$y$ mirror reflections.} 
\label{fig:trunc_examples}
\end{figure*}
%\section{Event localization algorithm}

\subsection{Image likelihood model}

We adopt a 2D Gaussian 
\begin{equation}
\mathcal{N}(\bm{t};\bm{\mu},\sigma^2)=\frac{1}{2\pi\sigma^2}\exp\left({-\frac{||\bm{t}-\bm{\mu}||_2^2}{2\sigma^2}}\right) 
\end{equation}
as the camera's point spread function and assume a circular Gaussian with 
covariance matrix $\Sigma=\sigma^2 I_{2\times2}$. 
$\sigma=ac$, where $a$ is a proportionality constant determined by the 
optical configuration, and $c$ is the circle of confusion diameter in \cref{eqn:circ_of_conf}.
$\bm{t}$ is a 2D coordinate on the sensor plane.

An event or mirror reflection located at $\bm{p_k}=(x_k,y_k,z_{ck})$ is modeled as 
a point source of light, so its image on the sensor consists of photon arrivals 
spatially distributed over a 2D Gaussian with mean 
\begin{equation} \label{eqn:mu}
\bm{\mu_k}=\left[ -\frac{S_2}{z_{ck}}x_k, -\frac{S_2}{z_{ck}}y_k \right]
\end{equation}
and standard deviation 
\begin{equation} \label{eqn:stdev}
\sigma_k=aA\frac{S_2}{S_1}\frac{|S_1-z_{ck}|}{z_{ck}}
\end{equation}
where \cref{eqn:mu} is derived from perspective projection.
Each photon arrival is one sample from this Gaussian.
%\textcolor{red}{
%In this paper, we flip the pixels of a frame so that photons arrive on the same 
%side of the sensor as the event (e.g. an event in the positive $x$-$y$ quadrant 
%produces an image on the sensor in the positive $x$-$y$ quadrant).
%This is the same as removing the negative signs in \cref{eqn:mu}.}

We model an image of an event in a kaleidoscopic scintillator as a GMM, where each 
component in the GMM corresponds to the event or a mirror reflection.
The complete-data likelihood function, $L$, for an image with $N$ photons, 
one event, and $K$ mirror reflections is
\begin{linenomath}
\begin{equation}
L(\bm{\theta};\bm{t},\bm{z})=\prod_{i=1}^N \prod_{k=0}^K \left[ \pi_k \mathcal{N}(\bm{t_i};\bm{\mu_k},\sigma_k^2) \right]^{\mathbbm{1}{(z_i=k)}}
\end{equation}
\end{linenomath}
where $\bm{\theta}=(\bm{\mu},\bm{\sigma},\bm{\pi})$ are model parameters, 
and $\pi_k$ is the mixing weight for component $k$.
$\bm{t}=(\bm{t_1}, \bm{t_2}, ..., \bm{t_N})$ are the 2D coordinates of each of 
$N$ photon arrivals on the sensor and $\bm{z}=(z_1, z_2, ..., z_N)$ are the latent 
variables of which component in the GMM that a photon belongs to.
We apply a density-based weighting scheme to photon samples to minimize the 
influence of sparsely distributed dark counts.
The weighted complete-data likelihood function, $L_w$, is
\begin{linenomath}
\begin{align}
L_w(\bm{\theta};\bm{t},\bm{z}) &= \prod_{i=1}^N \prod_{k=0}^K \left[ \pi_k \mathcal{N}(\bm{t_i};\bm{\mu_k},\frac{1}{w_i}\sigma_k^2) \right]^{\mathbbm{1}{(z_i=k)}} \\
&= \prod_{i=1}^N \prod_{k=0}^K \left[ \pi_k \frac{w_i}{2\pi{\sigma_k}^2} \text{exp}\left( -\frac{w_i}{2{\sigma_k}^2} ||\bm{t_i}-\bm{\mu_k}||_2^2 \right) \right]^{\mathbbm{1}{(z_i=k)}}
\end{align}
\end{linenomath}
where
\begin{linenomath}
\begin{align} \label{eqn:weights}
w_i = \sum_{j \in S_i^q} \text{exp} \left( -\frac{||\bm{t_i}-\bm{t_j}||_2^2}{\nu} \right)
\end{align}
\end{linenomath}
is the weight assigned to photon sample $i$, $S_i^q$ is the set of $q$ 
nearest neighbors of photon $i$, and $\nu$ is a positive scalar.
The expected value of the weighted complete-data log-likelihood, $Q$, is
\begin{linenomath}
\begin{equation} \label{eqn:Q_eqn}
\begin{aligned}
Q & = E_{\bm{z}|\bm{t}}\left[\log L_w(\bm{\theta};\bm{t},\bm{z})\right] \\ & = \sum_i \sum_k r_{ik} \left[ \text{log}(\pi_k) + \text{log}(w_i) - \text{log}(2\pi{\sigma_k}^2) - \frac{w_i}{2{\sigma_k}^2}||\bm{t_i}-\bm{\mu_k}||_2^2 \right]
\end{aligned}
\end{equation}
\end{linenomath}
where 
\begin{linenomath}
\begin{equation} \label{eqn:r_ik}
\begin{aligned}
r_{ik} & = E_{\bm{z}|\bm{t}}[\mathbbm{1}{(z_i=k)}] \\ & = \frac{\pi_k \mathcal{N}(\bm{t_i};\bm{\mu_k},{\sigma_k^2})}{\sum_{k'=0}^K \pi_{k'} \mathcal{N}(\bm{t_i};\bm{\mu_{k'}},{\sigma_{k'}^2})}
\end{aligned}
\end{equation}
\end{linenomath}
gives the posterior distribution of $\bm{z}$.
$r_{ik}$ is the probability that photon $i$ comes from event $k$, given 
the current parameter values.

The location of each mirror reflection at $\bm{p_k}$ for $k=1...K$ generated from 
an event at $\bm{p_0}=(x_0,y_0,z_{c0})$ is known based on the kaleidoscope's geometry.
For any mirror reflection $k$, each coordinate in $(x_k,y_k,z_{ck})$ is a linear 
combination of the event's $(x_0,y_0,z_{c0})$ coordinates based on the mirror's 
reflection transformation.
All $\bm{p_k}$'s can be written in terms of $(x_0,y_0,z_{c0})$ using 
\cref{eqn:ref_trans}, and all $\bm{\mu_k}$'s and $\sigma_k$'s can be written in 
terms of $(x_0,y_0,z_{c0})$ using \cref{eqn:mu,eqn:stdev}.
We can reduce the number of free parameters from $O(K)$ 
to $O(1)$ by rewriting each $\bm{\mu_k}$ and $\sigma_k$ in terms of 
$(x_0,y_0,z_{c0})$.
Thus, we obtain a GMM where each component is constrained to $\bm{p_0}$, and 
$\bm{\theta}=(x_0,y_0,z_{c0},\bm{\pi})$.
This results in an optimization problem for estimating $\bm{p_0}$ that captures 
the global information of the event and all mirror reflections:
\begin{linenomath}
\begin{equation} \label{eqn:opti_prob}
\argmax_{x_0,y_0,z_{c0},\bm{\pi}} Q
\end{equation}
\end{linenomath}
We optimize this using the EM algorithm.
$Q$ parameterized in terms of $\bm{p_0}$ is derived in Supplementary \cref*{sec:like,sec:weighted_like}.

\subsection{Optimization algorithm}
We assume a square pyramid kaleidoscope geometry, up to single-order reflections, 
and the presence of at least two mirror reflections in an image.
Each edge of the scintillator's square surface is parallel to each respective edge 
of the sensor.
In this configuration, mirror reflections will appear along either the 
$\pm x$ or $\pm y$ directions from the event's location.

One or more images of mirror reflections may be missing due to truncations 
depending on the event's location.
Therefore, determining which mirror reflections are present is required to compute $Q$.
We run the following initialization procedure to determine the presence of mirror 
reflections and to initialize the event's estimated location for the EM algorithm.

\noindent
\textbf{Initialization procedure.}
The initialization procedure can be summarized as computing 
\begin{equation}
\argmax_{x_0,y_0,z_{c0},C} Q
\end{equation}
over a set of possible event locations and $C \in \{3,4,5\}$ clusters 
(number of mirror reflections plus the event).
First, centroids are computed using weighted KMeans with $C$ clusters.
Centroids are then classified as either the event, or $+x$, $-x$, $+y$, or $-y$ 
mirror reflections based on their relative positioning.
This is done by taking combinations of subsets of centroids and computing the 
standard deviation of a subset's coordinates along the $x$ and $y$ dimensions to 
determine which centroids are horizontally or vertically aligned.
A set of possible event locations that is uniformly spaced over the depth 
($z$ dimension) of the scintillator is computed using the event's centroid and \cref{eqn:mu}.
The number of mirror reflections, which reflections are present, and the 
initialization point for $\bm{p_0}$ are those that correspond to the 
highest value of $Q$ out of this set and over $C \in \{3, 4, 5\}$.
When computing $Q$, terms that correspond to a mirror reflection $k$ are included
only if that mirror reflection is determined to be present.
$\bm{\pi}$ is initialized to the uniform distribution.
\iffalse
This initialization procedure is described in \cref{alg:init}.
$\bm{\pi}$ is initialized to the uniform distribution.


\begin{algorithm}
\caption{\textbf{Initialization procedure.}} \label{alg:init}
\DontPrintSemicolon
\SetKwInOut{Parameter}{Arguments}
\SetKwFunction{KMeans}{KMeans}
\SetKwFunction{ClassifyCentroids}{ClassifyCentroids}
\SetKwFunction{PossibleEventLocations}{PossibleEventLocations}
\SetKwFunction{Q}{Q}
\SetKwComment{Comment}{// }{}
%\begin{algorithmic}
\Parameter{photon locations $\bm{t}$, photon weights $\bm{w}$, lens to sensor distance $S_2$}
\KwOut{event initialization location $\bm{p_{init}}$, number of mirror reflections $K_{out}$, mirror reflection classification $M_{out}$}
$Q_{max} = -\infty$\;
\For{K=3:5}{
  $\bm{\mu_k}^0 \leftarrow \KMeans(\bm{t}, \bm{w}, K)$\;
  $M \leftarrow \ClassifyCentroids(\bm{\mu_k}^0)$\;
  $S \leftarrow \PossibleEventLocations(\bm{\mu_{event}}^0, S_2)$ \Comment{\cref{eqn:mu}}
  \For{$\bm{\tilde{p_0}} \in S$}{
    $Q_{\bm{\tilde{p_0}}} \leftarrow \Q(\bm{\tilde{p_0}})$\;
    \If{$Q_{\bm{\tilde{p_0}}} > Q_{max}$}{
      $Q_{max} \leftarrow Q_{\bm{\tilde{p_0}}}$\;
      $\bm{p_{init}} \leftarrow \bm{\tilde{p_0}}$\;
      $M_{out} \leftarrow M$\;
      $K_{out} \leftarrow K-1$\;
    }
  }
}
\end{algorithm}
\fi

\noindent
\textbf{Optimization procedure.}
During the E-step, $r_{ik}$ is updated using \cref{eqn:r_ik} and the current 
values of $\bm{p_0}$ and $\bm{\pi}$.
During the M-step, $\bm{p_0}$ is updated by fixing $\bm{\pi}$ and optimizing 
$Q$ with gradient ascent. 
$\bm{\pi}$ is then updated using $\pi_k=\frac{1}{N}\sum_{i=1}^N r_{ik}$.
Gradients are derived in Supplementary \cref*{sec:like,sec:weighted_like}.
In both the E and M steps, terms that correspond to a mirror reflection $k$ are 
included in the computation only if that mirror reflection is present in the image.
If truncation boundaries are known, then partial image truncations 
can be incorporated into the algorithm by zeroing $r_{ik}$ for photon
$i$ that lies in mirror $k$'s truncation zone.
We do not have accurate knowledge of truncation boundaries in experimental data,
so we only determine the presence of mirror reflections rather than partial 
truncations in the experiments.
The kaleidoscopic image model and event localization algorithm are validated on 
experimental data.
%\section{Simulations}

We use a square pyramid with a 20 mm wide base and 5.77 mm height for the 
kaleidoscopic scintillator's geometry.
The angle between opposite mirrors at the pyramid's apex is 120 degrees.
This wide angle geometry is chosen so that photons that undergo two or more 
reflections do not reach the camera and each mirror generates only one virtual event.
Thus, there is one real event and up to four virtual events in an image.

\subsection{Image simulation} \label{sec:image_sim}

We simulate the image of an event using the following procedure.
The scintillator is aligned with its base edges parallel to the sensor edges and centered.
We set the optical configuration parameters so that the field of view (FOV) and 
event image distributions approximately match what is observed in experimental 
data (\cref{sec:experiments}, \cref{fig:capture_results}) by visual inspection. 
This resulted in $S_1=60$ mm, $S_2=84$ mm, (focal length = 35 mm), $A=25$ mm, $a=0.5$.
For a given real event location $\bm{p_0}$, compute the virtual event location 
for each mirror using the mirror's reflection transformation.
For each event $k$, compute its corresponding $\bm{\mu_k}$ and $\sigma_k$ using \cref{eqn:mu,eqn:stdev}.
We set a minimum value of $\sigma_k=10$ pixels so that images are not focused 
onto one pixel when near the focal plane.
This minimum $\sigma_k$ is also incorporated in both the the grid search and EM algorithm.
We compute the truncation zone on the sensor for each virtual event.
For each event (real and virtual), draw 15 photon samples from $\mathcal{N}(\bm{\mu_k},\sigma_k)$ and assign each to the nearest pixel.
Discard samples that lie in the event's truncation zone or off the sensor.
If an image has less than 30 photons, do not evaluate this event location.
%We deem 15 samples per event before discarding to approximately match 
%experimentally observed counts, as shown in \cref{fig:capture_results}. 
We simulate a sensor of $1024 \times 1024$ pixels with 16 $\mu$m pixel pitch, 
which is the same as the hardware used in experiments.
Algorithms are tested on images with no dark counts and on the same images with 
added dark counts (noise).
For simulating noise, a number of dark counts is drawn from $Poisson(5)$ and 
randomly added to the image.


\subsection{Grid search}
We use a search grid of equispaced points consisting of 8 points in the 
$z$ dimension over $z_w=0.1$ mm to $z_w=5.25$ mm and 10 points in the $x$-$y$ 
dimension over $x=-4$ mm to $x=4$ mm and $y=-4$ mm to $y=4$ mm.
Points outside the kaleidoscope are not included.
We weight all events in an image equally and set $\bm{\pi}$ as a uniform 
distribution. $\bm{\pi}$ is not updated in the grid search algorithm.

We evaluate the grid search algorithm on two sets of grids.
In the ``aligned" set, all event locations lie exactly on a point in the grid search.
In the ``misaligned" set, the grid search is shifted so that event locations are  
halfway between adjacent grid search points.
Due to the kaleidoscope's symmetry, we limit the tested event locations to 
$x\geq0$ mm and $y\geq0$ mm.
We also evaluate the performance of incorporating truncations in the 
probabilistic model by running the grid search with and without setting 
$r_{ik}=0$ if a photon lies in the truncation zone.
We denote these two settings using ``trunc $r_{ik}$" and ``not trunc $r_{ik}$", 
respectively, in the results \cref{tab:sim_results_noise_and_no_noise}.


\subsection{EM algorithm}

The EM algorithm is tested on the same event locations and images as the grid 
search algorithm.
Optimizing $Q$ with gradient ascent is susceptible to local maxima because $Q$ 
is nonconvex.
For a given test location, we initialize the estimated event location,
$\bm{\tilde{p_0}}$, by randomly sampling a point $d_\text{init}$ mm away from the 
true location that lies within the scintillator.
We test two settings with $d_\text{init}=2,3$ mm.
We also test truncation modelling with $r_{ik}$ like in the grid search experiments.
The EM algorithm is terminated when the distance between the updated and 
previous $\bm{\tilde{p_0}}$ is less than $0.1$ mm or once 1000 iterations have occurred.

During one E-step, $r_{ik}$ is computed in \cref{eqn:r_ik} with the current 
values of $\bm{\tilde{p_0}}$ and $\bm{\pi}$ while incorporating truncations.

During one M-step, $\bm{\tilde{p_0}}$ is updated by optimizing 
\cref{eqn:opti_prob} using 1000 steps of gradient ascent.
The corresponding updates to $\bm{\tilde{p_0}}$ and $\bm{\pi}$ are
$\bm{\tilde{p_0}} := \bm{\tilde{p_0}} + \lambda \nabla_{\bm{p_0}} Q$ 
where $\lambda=1 \times 10^{-5}$
and $\pi_k := \frac{1}{N} \sum_{i=0}^N r_{ik}$.


\subsection{Simulation results}

We report the error between the predicted and true event locations, 
$d_{\text{err}}$, in terms of Euclidean distance. 
The test set consists of 77 event locations.
\cref{tab:sim_results_noise_and_no_noise} reports the frequencies of errors  
within a range for the grid search and EM algorithm.

% This is params used with test_grid_sensor5
\begin{table*}
\begin{tabular}{ |c|ccccc| } 
 \hline
 no noise | with noise  & $d_{\text{err}}=0$ mm & $0<d_{\text{err}}\leq1$ mm & $1<d_{\text{err}}\leq2$ mm & $2<d_{\text{err}}\leq3$ mm & $d_{\text{err}}>3$ mm \\ 
 \hline
 Grid search & & & & & \\
 aligned, trunc $r_{ik}$        & 77 | 61 & 0 | 1 & 0 | 7 & 0 | 5 & 0 | 3 \\ 
 aligned, not trunc $r_{ik}$    & 75 | 64  & 2 | 4 & 0 | 1 & 0 | 3 & 0 | 5 \\ 
 misaligned, trunc $r_{ik}$     & 0 | 0  & 75 | 58 & 2 | 9 & 0 | 5 & 0 | 5 \\
 misaligned, not trunc $r_{ik}$ & 0 | 0  & 76 | 65 & 0 | 6 & 1 | 2 & 0 | 4 \\
 \hline
 EM algorithm & & & & & \\
 $d_\text{init}=1$, trunc $r_{ik}$       & 0 | 0  & 67 | 42 & 10 | 20 & 0 | 5 & 0 | 10 \\
 $d_\text{init}=1$, not trunc $r_{ik}$   & 0 | 0  & 68 | 58 & 8 | 15 & 0 | 0 & 1 | 4 \\
 $d_\text{init}=2$, trunc $r_{ik}$       & 0 | 0  & 29 | 20 & 38 | 33 & 3 | 11 & 7 | 13 \\
 $d_\text{init}=2$, not trunc $r_{ik}$   & 0 | 0  & 44 | 34 & 25 | 28 & 5 | 5 & 3 | 10 \\
 $d_\text{init}=3$, trunc $r_{ik}$       & 0 | 0  & 12 | 6 & 36 | 25 & 19 | 13 & 10 | 33 \\
 $d_\text{init}=3$, not trunc $r_{ik}$   & 0 | 0  & 17 | 10 & 30 | 36 & 10 | 10 & 20 | 21 \\
 \hline
\end{tabular}
\caption{\textbf{Simulation localization error.} The frequencies of localization 
error ($d_{\text{err}}$) within a range for each algorithm and testing 
configuration. The left side and right sides of a column correspond to images 
without and with added dark counts, respectively.}
\label{tab:sim_results_noise_and_no_noise}
\end{table*}


\iffalse
% This uses the old gradient descent bugged gradient (but working well)
% This is params used with test_grid_sensor5
\begin{table*}
\begin{tabular}{ |c|ccccc| } 
 \hline
 no noise | with noise  & $d_{\text{err}}=0$ mm & $0<d_{\text{err}}\leq1$ mm & $1<d_{\text{err}}\leq2$ mm & $2<d_{\text{err}}\leq3$ mm & $d_{\text{err}}>3$ mm \\ 
 \hline
 Grid search & & & & & \\
 aligned, trunc $r_{ik}$        & 77 | 61 & 0 | 1 & 0 | 7 & 0 | 5 & 0 | 3 \\ 
 aligned, not trunc $r_{ik}$    & 75 | 64  & 2 | 4 & 0 | 1 & 0 | 3 & 0 | 5 \\ 
 misaligned, trunc $r_{ik}$     & 0 | 0  & 75 | 58 & 2 | 9 & 0 | 5 & 0 | 5 \\
 misaligned, not trunc $r_{ik}$ & 0 | 0  & 76 | 65 & 0 | 6 & 1 | 2 & 0 | 4 \\
 \hline
 EM algorithm & & & & & \\
 $d_\text{init}=2$, trunc $r_{ik}$       & 0 | 0  & 66 | 59 & 11 | 17 & 0 | 1 & 0 | 0 \\
 $d_\text{init}=2$, not trunc $r_{ik}$   & 0 | 0  & 52 | 53 & 20 | 18 & 0 | 1 & 5 | 5 \\
 $d_\text{init}=3$, trunc $r_{ik}$       & 0 | 0  & 45 | 45 & 29 | 28 & 3 | 3 & 0 | 1 \\
 $d_\text{init}=3$, not trunc $r_{ik}$   & 0 | 0  & 39 | 40 & 20 | 20 & 8 | 5 & 10 | 12 \\
 \hline
\end{tabular}
\caption{\textbf{Simulation localization error.} The frequencies of localization 
error ($d_{\text{err}}$) within a range for each algorithm and testing 
configuration. The left side and right sides of a column correspond to images 
without and with added dark counts, respectively.}
\label{tab:sim_results_noise_and_no_noise}
\end{table*}
\fi




\iffalse
% THIS IS PARAMS USED WITH test_grid_sensor3
\begin{table*}
\begin{tabular}{ |c|ccccc| } 
 \hline
 no noise | with noise  & $d_{\text{err}}=0$ mm & $0<d_{\text{err}}\leq1$ mm & $1<d_{\text{err}}\leq2$ mm & $2<d_{\text{err}}\leq3$ mm & $d_{\text{err}}>3$ mm \\ 
 \hline
 Grid search & & & & & \\
 aligned, trunc $r_{ik}$        & 115 | 42 & 9 | 10 & 5 | 21 & 0 | 24 & 0 | 32 \\ 
 aligned, not trunc $r_{ik}$    & 98 | 44  & 18 | 14 & 12 | 21 & 1 | 18 & 0 | 32 \\ 
 misaligned, trunc $r_{ik}$     & 0 | 0  & 122 | 35 & 4 | 32 & 3 | 22 & 0 | 40 \\
 misaligned, not trunc $r_{ik}$ & 0 | 0  & 117 | 46 & 10 | 28 & 1 | 18 & 1 | 37 \\
 \hline
 EM algorithm & & & & & \\
 $d_\text{init}=2$, trunc $r_{ik}$       & 0 | 0  & 90 | 58 & 39 | 46 & 0 | 6 & 0 | 19 \\
 $d_\text{init}=2$, not trunc $r_{ik}$   & 0 | 0  & 82 | 59 & 42 | 33 & 1 | 5 & 0 | 32 \\
 $d_\text{init}=3$, trunc $r_{ik}$       & 0 | 0  & 75 | 42 & 40 | 51 & 14 | 15 & 0 | 21 \\
 $d_\text{init}=3$, not trunc $r_{ik}$   & 0 | 0  & 63 | 43 & 35 | 28 & 21 | 24 & 10 | 34 \\
 \hline
\end{tabular}
\caption{\textbf{Simulation localization error.} The frequencies of localization 
error ($d_{\text{err}}$) within a range for each algorithm and testing 
configuration. The left side and right sides of a column correspond to images 
without and with added dark counts, respectively.}
\label{tab:sim_results_noise_and_no_noise_OLD}
\end{table*}
\fi



\iffalse
\begin{table*}
\begin{tabular}{ |c|ccccc| } 
 \hline
       & $d_{\text{err}}=0$ mm & $0<d_{\text{err}}\leq1$ mm & $1<d_{\text{err}}\leq2$ mm & $2<d_{\text{err}}\leq3$ mm & $d_{\text{err}}>3$ mm \\ 
 \hline
 Grid search & & & & & \\
 aligned, trunc $r_{ik}$        & 115 & 9   & 5  & 0 & 0 \\ 
 aligned, not trunc $r_{ik}$    & 98  & 18  & 12 & 1 & 0 \\ 
 misaligned, trunc $r_{ik}$     & 0   & 122 & 4  & 3 & 0 \\
 misaligned, not trunc $r_{ik}$ & 0   & 117 & 10 & 1 & 1 \\
 \hline
 EM algorithm & & & & & \\
 $d_\text{init}=2$, trunc $r_{ik}$       & 0   & 90 & 39 & 0 & 0 \\
 $d_\text{init}=2$, not trunc $r_{ik}$   & 0   & 82 & 42 & 1 & 0 \\
 $d_\text{init}=3$, trunc $r_{ik}$       & 0   & 75 & 40 & 14 & 0 \\
 $d_\text{init}=3$, not trunc $r_{ik}$   & 0   & 63 & 35 & 21 & 10 \\
 \hline
\end{tabular}
\caption{\textcolor{red}{ADD CAPTION!!!!!!!!! No dark counts in this table.}}
\label{tab:sim_results}
\end{table*}
\fi


\iffalse
\begin{table*}
\begin{tabular}{ |c|ccccc| } 
 \hline
       & $d_{\text{err}}=0$ mm & $0<d_{\text{err}}\leq1$ mm & $1<d_{\text{err}}\leq2$ mm & $2<d_{\text{err}}\leq3$ mm & $d_{\text{err}}>3$ mm \\ 
 \hline
 Grid search & & & & & \\
 aligned, trunc $r_{ik}$        & 42 & 10   & 21  & 24 & 32 \\ 
 aligned, not trunc $r_{ik}$    & 44  & 14  & 21 & 18 & 32 \\ 
 misaligned, trunc $r_{ik}$     & 0   & 35 & 32  & 22 & 40 \\
 misaligned, not trunc $r_{ik}$ & 0   & 46 & 28 & 18 & 37 \\
 \hline
 EM algorithm & & & & & \\
 $d_\text{init}=2$, trunc $r_{ik}$       & 0   & 58 & 46 & 6 & 19 \\
 $d_\text{init}=2$, not trunc $r_{ik}$   & 0   & 59 & 33 & 5 & 32 \\
 $d_\text{init}=3$, trunc $r_{ik}$       & 0   & 42 & 51 & 15 & 21 \\
 $d_\text{init}=3$, not trunc $r_{ik}$   & 0   & 43 & 28 & 24 & 34 \\
 \hline
\end{tabular}
\caption{\textcolor{red}{ADD CAPTION!!!!!!!!! Dark counts are in this table.}}
\label{tab:sim_results_dark_counts}
\end{table*}
\fi
%\section{Experiments}
\textbf{Hardware and data collection.}
The experimental hardware consists of a SPAD array, lens, scintillator, and 1 $\mu$Ci Co-60 gamma-ray source.
We use the SPAD512 array (Pi Imaging) with microlenses for increased fill factor, 
which has $512 \times 512$ pixels and 16 $\mu$m pixel pitch.
The SPAD array is configured to capture 1-bit images with 1.5 $\mu$s 
integration time.
The lens is a 50 mm focal length Nikkor lens set to a f/1.2 aperture.
The scintillator is a GAGG(Ce)-HL crystal (Epic Crystal), which has a 150 ns decay 
constant, a 530 nm emission peak, and an index of refraction of 1.91.
Its geometry is a square pyramid with a 20 mm wide base and 5.77 mm 
height and a 120 degree opening angle at the apex.
Four surfaces of the scintillator are coated with enhanced specular reflector.
We account for the scintillator's index of refraction by assuming the 
scintillator's height is $5.77$ mm  / 1.91 $=3.02$ mm and compute apparent event 
locations in the smaller volume.
The camera's lateral field of view (FOV) covers approximately $5 \times 5$ mm at 
the focal plane.
The scintillator is positioned such that its apex is in-focus and centered in the 
FOV, and that its corners are aligned with the corners of the FOV. 
There is an air gap of approximately 30 mm between the scintillator and the lens.
The entire setup is placed inside a light-tight box to keep ambient light out.
The scintillator is shown in Supplementary Fig. \ref*{fig:scintillator}.
The experimental camera focus on the scintillator is shown in Supplementary Fig. \ref*{fig:experiment_focus}. 

Data collection took place at about 21 degrees Celsius ambient temperature.
We first captured 130,000 images without the gamma-ray source to characterize the 
dark count rate and zero out 5\% of pixels with the highest dark count rates.
After zeroing 5\% of pixels, we observe a median of 4 dark counts per image.
A histogram of dark counts per image is shown in Supplementary Fig. \ref*{fig:dark_counts_hist}.

We collect data by placing the gamma-ray source adjacent to the scintillator's 
apex and passively capturing images.
All computations are performed in post-processing.
Zeroing 5\% of pixels is applied to all images.
We capture 13,000,000 images with the gamma-ray source present and 
discard images with less than 60 counts.
Histograms of counts in an image are shown in Supplementary \cref*{fig:cap_counts_hist}.

\noindent
\textbf{Algorithm parameter values and stopping criteria.}
We use $q=10$ nearest neighbors and $\nu=10$ pixels for assigning photon weights 
using \cref{eqn:weights}.
Up to ten possible event locations are evaluated in the initialization procedure 
equispaced over the scintillator's depth.
During one M-step, we run gradient ascent for 1,000 steps with a step size of 1e-7.
We run the EM algorithm until the distance in the estimated event location between 
consecutive steps is less than 0.01 mm, or until 100 steps are taken.

\noindent
\textbf{Experimental camera parameter calibration.}
Experimental camera parameter values for $S_1$, $S_2$, and $a$ are calibrated by 
optimizing $Q$ in a grid search on a selected experimental image.
We select the image shown in Supplementary \cref*{fig:calibration}, where the 
event is approximately centered with evident mirror reflections, and we manually 
remove dark counts.
The grid search is performed over event locations at $x$-$y$ coordinates 
$(0,0)$ mm and $z$-coordinates that span the depth of the scintillator.
This resulted in $S_1=37.24$ mm, $S_2=109.48$ mm, and $a=0.056$ at event location 
(0, 0, 2.61) mm.

\noindent
\textbf{Kaleidoscopic image model validation.}
We select six experimental images and overlay the algorithm's estimated Gaussian 
components, shown in \cref{fig:example_figures}, to validate the presence of 
mirror reflections in accordance with the kaleidoscopic model.
Due to non-idealities, truncation zone boundaries in experimental images do not 
exactly match those derived theoretically using a thin lens.
We do not attempt to derive truncation lines in experimental images.
Rather, truncations are evident from missing mirror reflections in the selected images.

\begin{figure*}
\centering
\includegraphics[width=\linewidth]{example_figures.pdf}
\caption{\textbf{Selected experimental images.} Experimental images overlaid with the algorithm's estimated Gaussians.
Each dashed red circle is centered on the Gaussian component's mean. 
The inner and outer circles are one and two standard deviations in radius, respectively.
Pixels with a photon are enlarged with a $3 \times 3$ filter for visualization purposes.
} 
\label{fig:example_figures}
\end{figure*}

\noindent
\textbf{Algorithm validation.}
Experimental events cannot be controlled, so their ground truth locations are unknown.
Therefore, to validate the algorithm is locating the event, we report agreement of 
multiple measurements of the event's location as follows.
We use experimental images that contain the event and four mirror reflections as 
test images.
The number of mirror reflections in an image is determined using the algorithm's 
initialization procedure.
Then, we create new images of the event by removing combinations of one or two 
mirror reflections from the image.
Photon $i$ is classified as belonging to mirror reflection $k$ according to 
$\max_k r_{ik}$ using $r_{ik}$ values obtained after running the algorithm's 
optimization procedure on the original test image with four mirror reflections.
Thus, for one test image, we generate 4 images with one mirror reflection removed 
and 6 images with two mirror reflections removed for a total of 11 images 
corresponding to one event.
We run the algorithm's initialization and optimization procedures on each image to 
obtain multiple measurements of the event's location.
We compute the mean estimated event location over the 11 images corresponding to 
one event.
We record the distance between the mean location and the estimated location for 
each individual image.
The distribution of this distance over all test images is used to report the 
agreement in event location measurements, where short distances indicate good 
agreement (high precision).
Selecting images that contain four mirror reflections and at least 60 counts 
resulted in 2,251 test images.
Recorded distances are reported in histograms in \cref{fig:crossval_error}.
%Statistics on the distances in Group A over different ranges of photon counts in a 
%test image are reported in \cref{tab:crossval_thresh_error}.

We test for the algorithm's convergence by using two different initialization 
methods for $\bm{p_0}$. 
We either initialize $\bm{p_0}$ adaptively to the image as in the algorithm's 
initialization procedure or to $(0,0,1.5)$ mm.
We refer to the two initializations as ``regular" and ``fixed", respectively.
The test images for this convergence experiment consist of all experimental images 
that contain at least 60 counts, for a total of 4,351 images.
Distances between estimated event locations using the regular and fixed 
initialization methods, as well as the number of EM steps taken, are shown in \cref{fig:convergence}.

\begin{figure}
\centering
\includegraphics[width=\linewidth]{crossval_error_v2.pdf}
\caption{\textbf{Agreement in event location measurements.} Histogram of distances between mean estimated event location and each image's estimated event location after mirror reflection removals. Distances including both single and double mirror reflection removals. Mean $\pm$ stdev: 0.24 $\pm$ 0.34 mm, 24,761 distances.
} 
\label{fig:crossval_error}
\end{figure}

\begin{figure*}
\centering
\includegraphics[width=\linewidth]{fixed_init_results.pdf}
\caption{\textbf{Optimization convergence.}
(a) Distances in estimated event locations between the regular and fixed initialization methods. Mean $\pm$ stdev: $0.38 \pm 0.80$ mm, 4,351 events.
(b) Number of steps taken in the EM algorithm from the regular initialization point. Mean $\pm$ stdev: $9.3 \pm 14.2$ steps, 4,351 events.
(c) Number of steps taken in the EM algorithm from the fixed initialization point. Mean $\pm$ stdev: $13.5 \pm 14.9$ steps, 4,351 events.
} 
\label{fig:convergence}
\end{figure*}


\iffalse
\begin{table}[h!]
\centering
\begin{tabular}{|c|ccccc|}
\hline
Counts in a test image & 60 - 79 & 80 - 99 & 100 - 119 & 120 - 139 & 140 - 159 \\
\hline
Mean (mm)                & 0.28 & 0.26 & 0.19 & 0.19 & 0.15 \\
Standard dev.  (mm)      & 0.36 & 0.35 & 0.30 & 0.29 & 0.21  \\
Number of distances    & 9,317  & 7,524  & 4,774  & 2,354  & 792 \\
\hline
\end{tabular}
\caption{\textbf{Agreement in event location measurements.} Mean and standard 
deviation of distances between mean estimated event location and each image's 
estimated event location after both single and double mirror reflection removals 
(Group A) reported over different ranges of counts in a test image.}
\label{tab:crossval_thresh_error}
\end{table}
\fi


\section{Discussion}

One kaleidoscopic geometry is tested in this work with a wide opening angle at the 
apex so that imaging second-order reflections is rare or impossible. 
However, the kaleidoscope is configurable in terms of number of surfaces, opening 
angle, and size.
For example, choosing a smaller opening angle will increase the number of mirror 
reflections and light collection at the cost of a smaller scintillator volume and 
higher-complexity images with less distance between mirror reflections.
Dividing the scintillator into more surfaces would increase the number of mirror 
reflections but also the amount of image truncation.
Overall, the kaleidoscopic scintillator can increase imaged photon counts by 
multiple factors for increased energy resolution and provide multiple views for 
measuring events with increased accuracy and robustness using a single camera.

\noindent
\textbf{Simulated validation of theoretical image truncations.}
The mirror reflections generated with ray tracing and a thin lens are truncated 
along the theoretically derived acceptance zones, shown in \cref{fig:trunc_examples}. 
This suggests the theoretical truncation model is correct.
In \cref{fig:trunc_examples}, the +$x$, -$x$, and -$y$ mirror reflections are 
partially truncated, while the +$y$ mirror reflection is not truncated.
Obtaining accurate experimental acceptance zones would require camera 
calibration for lens distortions, adapting the scintillator's edges accordingly, 
and precise knowledge of the focal plane.

\noindent
\textbf{Experimental validation of kaleidoscopic image model.}
Various examples where the Gaussian components predicted by the 
algorithm approximately overlap the photon clusters of the event and mirror 
reflections by visual inspection are shown in \cref{fig:example_figures}.
This suggests that mirror reflections are indeed being captured in accordance with 
the proposed model.
Examples also demonstrate the algorithm can correctly determine the number of 
mirror reflections present in the image.
\cref{fig:example_figures}a,b are examples where an event occurred at high and 
low $z_w$-coordinates, respectively.
\cref{fig:example_figures}c,d illustrate cases where a mirror reflection lies 
outside the field of view. 
\cref{fig:example_figures}e,f show examples where mirror reflections would lie in 
the field of view but are completely truncated.
\cref{fig:example_figures}e illustrates a case where the +$y$ mirror reflection is 
completely truncated, and \cref{fig:example_figures}f illustrates a case where the 
-$x$ and -$y$ mirror reflections are completely truncated.
Dark counts in \cref{fig:example_figures} are higher than the median dark count 
rate of 4 pixels per image.
This could be due to several factors, including increased cross talk from higher 
photon collection, fluorescence from other particles or gamma-rays occurring at 
the beginning or end of an image's integration time, imperfect mirror reflections, 
or internal reflections.

Internal reflections may be contributing to background noise in addition to dark counts.
However, internal reflections at the scintillator's base surface could possibly 
appear in an image as another mirror reflection of the event.
An internal reflection has the same effect as adding a mirror to the 
scintillator's base surface, which can be modeled with the theory introduced in 
this paper.
Since internal reflection occurs at a large incidence angle, the 
mirror reflection of an internal reflection will likely exist 
outward from the optical axis away from mirror reflections formed 
without internal reflection. 
The camera's FOV in the experiments is limited so that we do not 
expect to image mirror reflections that follow from internal reflections.
Imaging internal reflections will increase light collection but complicate the 
captured image.

\noindent
\textbf{Experimental validation of localization algorithm.}
The results in \cref{fig:crossval_error} demonstrate an agreement in event 
location measurements among combinations of mirror reflection removals.
This agreement suggests the algorithm is in fact estimating the event's location, 
and that mirror reflections provide robustness in addition to increased photon counts.

The mean distance in event location estimates between regular and fixed 
initialization points is small, shown in \cref{fig:convergence}a.
This indicates that the gradient ascent optimization during the M-step does in 
fact step toward the event location and approaches it.
\cref{fig:convergence}b,c indicate that setting the initialization point using the 
algorithm's initialization proecdure results in convergence with fewer EM steps 
because of its proximity to the event compared to a fixed initialization point. 
Slightly elevated frequency in convergence distances between 1 and 2 mm shown in 
\cref{fig:convergence}a indicates that a fixed initialization can result in 
convergence at a local optimum rather than the event location.

\noindent
\textbf{Limitations and future work.}
A goal of this work is to validate that our proposed localization algorithm is in 
fact locating the event.
We do so by reporting that the algorithm obtains spatially close estimates of an 
event's location over multiple images of the event after removing mirror reflections.
This provides a measure of localization precision.
Performing a more complete characterization of the experimental system, 
including event localization accuracy and resolution, 
would require scanning the scintillator with a collimated gamma-ray source in
order to obtain a form of ground truth event locations.
This work is meant to demonstrate a new radiation detector concept.
Future work may optimize the hardware configuration or extend the algorithm for 
specific downstream tasks and characterize overall performance more thoroughly.
Further investigation of localization accuracy for this specific experimental 
configuration would provide little additional insight at this stage.

While this work considers one event per image, multiple events from a single 
particle or gamma-ray can occur. 
Measuring multi-event cases is needed to perform advanced radiation imaging 
techniques such as in the Compton camera or neutron scatter camera.
Future work may consider how to model images that contain multiple events in a 
kaleidoscopic scintillator.

\textcolor{red}{The calibration procedure resulted in Gaussian mixture model (GMM) components 
whose centroids are close to but not exactly centered on the mirror reflections in 
the calibration image in \cref{fig:calibration}.
Similarly in \cref{fig:example_figures}c, the Gaussian component's centroid is 
slightly offset from the -$y$ mirror reflection.
This is likely due to $Q$'s definition and lens distortion.
An event's location influences both the mean and standard deviation of its image.
$Q$ takes both parameters into account jointly over the event and all mirror 
reflections, so the optimal $Q$ value may not correspond to components in the GMM 
that are all centered on the event or mirror reflections.
Lens distortions may also cause shifts in the event and mirror reflections that 
are not accounted for in the model.
Nonetheless, GMM components are overlaid accurately on the event and mirror 
reflections in \cref{fig:example_figures}a,b,d,e,f with the same calibrated parameters.}

\iffalse
\textcolor{red}{
While this work models one event, multiple events from a single particle or 
gamma-ray can occur. 
For example, a neutron or gamma-ray can scatter one or more times before exiting 
or being absorbed in the scintillator.
Measuring a pair of scatter-absorption events from a single particle forms the 
basis for backprojection to determine the particle's trajectory and the radiation 
source's location.
In general, the ability to measure multiple, simultaneous events allows to 
accurately measure how and where a particle interacted in the scintillator.
Adequate energy, spatial, and time resolutions are required to perform the Compton backprojection.
SPAD camera designs have sufficient spatial and time resolutions, but 
they are limited by low light collection.
Now, the kaleidoscopic scintillator may provide the light collection levels needed 
to perform these advanced techniques with a SPAD camera.
Future work may consider how to model images that contain multiple events in a 
kaleidoscopic scintillator.
}
\fi

\section{Methods} \label{sec:methods}

\subsection{Kaleidoscopic event imaging theory} \label{sec:theory}

Without loss of generality, consider a square pyramid kaleidoscope where each 
face is a specular surface, 
and the index of refraction inside the kaleidoscope is $n=1$.
The base of the pyramid is open, through which light exits the kaleidoscope.
Corrections for modeling a kaleidoscopic scintillator with $n>1$ include 
a decreased pyramid height and increased opening angle at the apex, based on $n$.
The true depth of a measured event would then be corrected for using $n$.
The image of a scintillation event consists of light emitted directly from the 
event and from the event's mirror reflections.
Below, we describe the spatial relationship between an event and its mirror 
reflections, as well as their image on the sensor using a thin lens model.


\subsubsection{Imaging configuration}

The world coordinate system's origin is set at the pyramid's apex with the 
$z$-axis directed perpendicular toward the pyramid's base surface.
The camera coordinate system's origin is at the center of the lens with its 
$z$-axis directed toward the pyramid in the opposite direction of the world's $z$-axis.
The $x$ and $y$ axes among the two coordinate systems are in the same directions, 
so transforming a point between the world and camera coordinate systems is 
carried out by adding or subtracting the $z$-coordinate with the distance between 
the lens and the scintillator's apex.
We denote a $z$-coordinate in the world coordinate system as $z_w$ and camera 
coordinate system as $z_c$.
We use a thin lens to model the camera.
Three important planes to note are the focal plane, the thin lens plane, and the 
sensor plane.
The focal plane is set at the pyramid's apex at $z_w=0$ mm.
A thin lens with diameter $A$ is placed at a distance $S_1$ from the focal plane. 
The sensor is placed at a distance $S_2$ from the lens.
The lens' focal length is set to $f=(S_1^{-1}+S_2^{-1})^{-1}$.
These parameters are illustrated in \cref{fig:optical_config}.

\ifthenelse{\boolean{figs_in_text}}{

\begin{figure}
\centering
\includegraphics[width=.33\linewidth]{optical_config.pdf}
\caption{\textbf{Imaging parameters and coordinate systems.} 
The optical configuration consists of a lens with diameter $A$, distance from 
lens to focal plane $S_1$, and distance from lens to sensor $S_2$. The diameter 
of an event's image on the sensor is $c$. The camera is focused at the 
scintillator's apex where the world coordinate system origin is located. The 
camera coordinate system origin is located at the lens. The figure only shows 
light emitted directly to the camera.} 
\label{fig:optical_config}
\end{figure}

}{}

A scintillation event is approximated as a point source of light.
Since the optical setup is constrained to short imaging distances, 
the images of an event and its mirror reflections exhibit defocus blur and 
have nonzero diameters.
An image's diameter on the sensor varies according to the event's distance from 
the focal plane, following the circle of confusion model.
For an event at $(x_0,y_0,z_{c0})$, 
the circle of confusion model yields
\begin{equation} \label{eqn:circ_of_conf}
c=A\frac{S_2}{S_1}\frac{|S_1-z_{c0}|}{z_{c0}}
\end{equation}
where $c$ is the image diameter at the sensor.

Light is emitted in all directions from the event.
Photons may arrive at the camera directly from the event or indirectly after 
reflecting off mirrors.
Direct photons form a complete image on the sensor, while indirect photons may 
form an image that is truncated or completely missing as described below.


\subsubsection{Mirror reflections and apertures}

Consider an event at $\bm{p_0}=(x_0,y_0,z_{w0})$.
Mirror $k$ with normal vector $\bm{n_k}$ produces a mirror reflection located at 
\begin{equation}
\bm{p_k}=T_k\bm{p_0}
\end{equation}
where
\begin{equation} \label{eqn:ref_trans}
T_k=I_{3\times3} - 2\bm{n_k}\bm{n_k}^T
\end{equation}
is the mirror's transformation.
The mirror reflection of an event is also a point source of light.
The captured image of the mirror reflection is obtained from the photons that 
reflect off the mirror and into the camera, exhibiting the same defocus blur as if 
an event were located at $\bm{p_k}$.
However, due to the finite mirror size, the image on the sensor may be truncated 
along lines corresponding to the mirror's edges.
This occurs when $\bm{p_k}$ is behind another mirror from the camera's perspective.
The photons that are truncated from the image are those that reflect off the 
mirror adjacent to mirror $k$ near the shared edge.
Also, light in higher-order reflections over multiple mirrors may be stopped 
in previous reflections and also cause image truncations.
Essentially, mirror $k$ behaves like an aperture to a light source at $\bm{p_k}$. 

For a single reflection, light from $\bm{p_k}$ that does not pass through mirror 
$k$ does not reach the camera and is truncated from the image.
\cref{fig:trunc_theory}a and c show a 2D view of the propagation of light over a single reflection without truncations.
\cref{fig:trunc_theory}b and d show how truncations form.
A 3D visualization of light truncation is shown in \cref{fig:trunc_teaser}.


\ifthenelse{\boolean{figs_in_text}}{

\begin{figure*}
\centering
\includegraphics[width=\linewidth]{trunc_theory.pdf}
\caption{\textbf{Mirror apertures and image truncations.} 
An event emits light onto a mirror that reflects into the camera. 
Some light might not reach the sensor due to finite mirrors and defocus blur.
(a) The mirror reflection is located beyond the focal plane. 
All light that forms the mirror reflection reaches the sensor.
(b) The mirror reflection is located beyond the focal plane. 
Some light from the mirror reflection is stopped at the mirror's edge and 
truncated on the sensor.
(c) The mirror reflection is located within the focal plane.
All light that forms the mirror reflection reaches the sensor.
(d) The mirror reflection is located within the focal plane. 
Some light from the mirror reflection is stopped at the mirror's edge and 
truncated on the sensor.
(e) Light from a double mirror reflection is stopped at both mirrors' edges and truncated on the sensor. 
The mirror for the second reflection is illustrated along the optical axis.} 
\label{fig:trunc_theory}
\end{figure*}


\begin{figure}
\centering
\includegraphics[width=.4\linewidth]{trunc_teaser.jpg}
\caption{\textbf{Image truncation example in 3D.}
The image of an event and one mirror reflection is shown. 
Light corresponding to the mirror reflection is truncated at the mirror's edge, 
resulting in a truncation line in the image.
The truncation line only applies to that mirror reflection.}
\label{fig:trunc_teaser}
\end{figure}

}{}


In higher-order reflections, light reflecting off mirror $k$ can 
reflect off another mirror $l$ and generate mirror reflection 
at $\bm{p_l}=T_l \bm{p_k}$.
The light incident on mirror $l$ from $\bm{p_k}$ passes through the aperture 
of mirror $k$.
If this light spans a partial area $A_l$ of mirror $l$, then mirror $l$'s aperture is $A_l$.
Otherwise, mirror $l$'s aperture is simply the mirror.
Mirror $l$'s aperture affects the light emitted from $\bm{p_l}$ toward another 
mirror for a higher-order reflection and toward the camera for imaging.
Higher-order reflections and imaging continue in the same manner.
A 2D view of truncations from multiple reflections is shown in \cref{fig:trunc_theory}e.


\subsubsection{Image truncations} \label{sec:image_truncations}

Consider an event's mirror reflection located at $\bm{p_k}$ generated from mirror $k$.
Each edge of the mirror may impose a truncation line on the camera sensor.
A truncation line on the sensor, $l_\text{sensor}$, is determined as follows.
Denote a truncation plane, $P_{\text{trunc}}$, that contains $\bm{p_k}$ and the 
mirror's edge.
$P_{\text{trunc}}$ blocks light emitted toward the side facing away from the mirror.
Denote this side using the normal vector $\bm{n_{\text{trunc}}}$.
Compute the intersection between $P_{\text{trunc}}$ and the focal plane to be the
truncation line at the focal plane, $l_\text{focal}$.
Project $\bm{n_{\text{trunc}}}$ onto the focal plane, denoted as 
$\bm{n_{\text{focal}}}$.
Scale $l_\text{focal}$ by the magnification, $m=-\frac{S2}{S1}$, to obtain $l_\text{sensor}$.
Scale $\bm{n_{\text{focal}}}$ by $m$ to obtain $\bm{n_{\text{sensor}}}$.

The truncation side of $l_\text{sensor}$ where photons do not arrive depends on 
which side of the focal plane that $\bm{p_k}$ is located.
If $\bm{p_k}$ is beyond the focal plane away from the lens at a distance $d>S_1$ 
from the lens, 
then the side of $l_\text{sensor}$ pointed to by $\bm{n_{\text{sensor}}}$ is 
truncated and contains no photon arrivals (\cref{fig:trunc_theory}b).
If $\bm{p_k}$ is between the focal plane and lens at a distance $d<S_1$ from the lens, 
then the side of $l_\text{sensor}$ opposite of $\bm{n_{\text{sensor}}}$ 
is truncated and contains no photon arrivals (\cref{fig:trunc_theory}d).

The area on the sensor where photons cannot arrive is the union of the
truncation sides of each truncation line.
We denote this area as the ``truncation zone" and its complement as the 
``acceptance zone" (\cref{fig:trunc_theory}b,d,e).
The event itself has no truncation zone on the sensor because its image is formed 
by light emitted directly to the camera without reflections.


\subsection{Event localization algorithm}

We adopt a 2D Gaussian 
\begin{equation}
\mathcal{N}(\bm{t};\bm{\mu},\sigma^2)=\frac{1}{2\pi\sigma^2}\exp\left({-\frac{(\bm{t}-\bm{\mu})^T(\bm{t}-\bm{\mu})}{2\sigma^2}}\right) 
\end{equation}
as the camera's point spread function and assume a circular Gaussian with 
covariance matrix $\Sigma=\sigma^2 I_{2\times2}$. 
$\sigma=ac$, where $a$ is a proportionality constant determined by the 
optical configuration, and $c$ is the circle of confusion diameter in \cref{eqn:circ_of_conf}.
$\bm{t}$ is a 2D coordinate on the sensor plane.

An event or mirror reflection located at $\bm{p_k}=(x_k,y_k,z_{ck})$ is modeled as 
a point source of light, so its image on the sensor consists of photon arrivals 
spatially distributed over a 2D Gaussian with mean 
\begin{equation} \label{eqn:mu}
\bm{\mu_k}=\left[ -\frac{S_2}{z_{ck}}x_k, -\frac{S_2}{z_{ck}}y_k \right]
\end{equation}
and standard deviation 
\begin{equation} \label{eqn:stdev}
\sigma=aA\frac{S_2}{S_1}\frac{|S_1-z_{ck}|}{z_{ck}}
\end{equation}
where \cref{eqn:mu} is derived from perspective projection.
Each photon arrival is one sample from this Gaussian.
%\textcolor{red}{
%In this paper, we flip the pixels of a frame so that photons arrive on the same 
%side of the sensor as the event (e.g. an event in the positive $x$-$y$ quadrant 
%produces an image on the sensor in the positive $x$-$y$ quadrant).
%This is the same as removing the negative signs in \cref{eqn:mu}.}

We model an image of an event in a kaleidoscopic scintillator as a GMM, where each 
component in the GMM corresponds to the event or a mirror reflection.
The complete-data likelihood function, $L$, for an image with $N$ photons, 
one event, and $K$ mirror reflections is
\begin{linenomath}
\begin{equation}
L(\bm{\theta};\bm{t},\bm{z})=\prod_{i=1}^N \prod_{k=0}^K \left[ \pi_k \mathcal{N}(\bm{t_i};\bm{\mu_k},\sigma_k^2) \right]^{\mathbbm{1}{(z_i=k)}}
\end{equation}
\end{linenomath}
where $\bm{\theta}=(\bm{\mu},\bm{\sigma},\bm{\pi})$ are model parameters, 
and $\pi_k=P(z_i=k)$ is the prior distribution that a photon comes from 
component $k$.
$\bm{t}=(\bm{t_1}, \bm{t_2}, ..., \bm{t_N})$ are the 2D coordinates of each of 
$N$ photon arrivals on the sensor and $\bm{z}=(z_1, z_2, ..., z_N)$ are the latent 
variables of which component in the GMM that a photon belongs to.
We apply a density-based weighting scheme to photon samples to minimize the 
influence of sparsely distributed dark counts.
The weighted complete-data likelihood function, $L_w$, is
\begin{linenomath}
\begin{align}
L_w(\bm{\theta};\bm{t},\bm{z}) &= \prod_{i=1}^N \prod_{k=0}^K \left[ \pi_k \mathcal{N}(\bm{t_i};\bm{\mu_k},\frac{1}{w_i}\sigma_k^2) \right]^{\mathbbm{1}{(z_i=k)}} \\
&= \prod_{i=1}^N \prod_{k=0}^K \left[ \pi_k \frac{w_i}{2\pi{\sigma_k}^2} \text{exp}\left( -\frac{w_i}{2{\sigma_k}^2} (\bm{t_i}-\bm{\mu_k})^T(\bm{t_i}-\bm{\mu_k}) \right) \right]^{\mathbbm{1}{(z_i=k)}}
\end{align}
\end{linenomath}
where
\begin{linenomath}
\begin{align} \label{eqn:weights}
w_i = \sum_{j \in S_i^q} \text{exp} \left( -\frac{||\bm{t_i}-\bm{t_j}||_2^2}{\nu} \right)
\end{align}
\end{linenomath}
is the weight assigned to photon sample $i$, $S_i^q$ is the set of $q$ 
nearest neighbors of photon $i$, and $\nu$ is a positive scalar.
The expected value of the weighted complete-data log-likelihood, $Q$, is
\begin{linenomath}
\begin{equation} \label{eqn:Q_eqn}
\begin{aligned}
Q & = E_{\bm{z}|\bm{t}}\left[\log L_w(\bm{\theta};\bm{t},\bm{z})\right] \\ & = \sum_i \sum_k r_{ik} \left[ \text{log}(\pi_k) + \text{log}(w_i) - \text{log}(2\pi{\sigma_k}^2) - \frac{w_i{\sigma_k}^{-2}}{2}(\bm{t_i}-\bm{\mu_k})^T(\bm{t_i}-\bm{\mu_k}) \right]
\end{aligned}
\end{equation}
\end{linenomath}
where 
\begin{linenomath}
\begin{equation} \label{eqn:r_ik}
\begin{aligned}
r_{ik} & = E_{\bm{z}|\bm{t}}[\mathbbm{1}{(z_i=k)}] \\ & = \frac{\pi_k \mathcal{N}(\bm{t_i};\bm{\mu_k},{\sigma_k^2})}{\sum_{k'=0}^K \pi_{k'} \mathcal{N}(\bm{t_i};\bm{\mu_{k'}},{\sigma_{k'}^2})}
\end{aligned}
\end{equation}
\end{linenomath}
gives the posterior distribution of $\bm{z}$.
$r_{ik}$ is the probability that photon $i$ comes from event $k$, given 
the current parameter values.

Truncations can be incorporated in setting the value of $r_{ik}$. 
If photon $i$ at location $\bm{t_i}$ on the sensor lies in the truncation zone 
belonging to mirror reflection $k$, then the probability that photon 
$i$ belongs to mirror reflection $k$ is 0. 
In this case, $r_{ik}=0$, and the corresponding term in the 
summation in the denominator of \cref{eqn:r_ik} is 0 when solving for a different $r_{ik}$.

The location of each mirror reflection at $\bm{p_k}$ for $k=1...K$ generated from 
an event at $\bm{p_0}$ is known based on the kaleidoscope's geometry.
For any mirror reflection $k$, each coordinate in $(x_k,y_k,z_{ck})$ is a linear 
combination of the event's $(x_0,y_0,z_{c0})$ coordinates based on the mirror's 
reflection transformation.
All $\bm{p_k}$'s can be written in terms of $(x_0,y_0,z_{c0})$ using 
\cref{eqn:ref_trans}, and all $\bm{\mu_k}$'s and $\sigma_k$'s can be written in 
terms of $(x_0,y_0,z_{c0})$ using \cref{eqn:mu,eqn:stdev}.
We can reduce the number of free parameters from $O(K)$ 
to $O(1)$ by rewriting each $\bm{\mu_k}$ and $\sigma_k$ in terms of 
$(x_0,y_0,z_{c0})$.
Thus, we obtain a GMM where each component is constrained to $\bm{p_0}$, and 
$\bm{\theta}=(x_0,y_0,z_{c0},\bm{\pi})$.
This results in an optimization problem for estimating $\bm{p_0}$ that captures 
the global information of the event and all mirror reflections:
\begin{linenomath}
\begin{equation} \label{eqn:opti_prob}
\argmax_{x_0,y_0,z_{c0},\bm{\pi}} Q
\end{equation}
\end{linenomath}
We optimize this using the EM algorithm.
$Q$ parameterized in terms of $\bm{p_0}$ is derived in Supplementary \cref*{sec:like,sec:weighted_like}.

One or more images of mirror reflections may be missing due to truncations, 
depending on the event's location.
Therefore, determining which mirror reflections are present is required to compute $Q$.
We run the following initialization procedure to determine the presence of mirror 
reflections and to initialize the event's estimated location for the EM algorithm.

We assume a square pyramid kaleidoscope geometry, up to single-order reflections, 
and the presence of at least two mirror reflections in an image.
Each side of the scintillator's square surface is parallel to each respective side 
of the sensor.
Mirror reflections will appear along either the $x$ or $y$ axes from the event.
In the initialization procedure, centroids $\bm{\mu_k}^0$ are obtained using 
weighted KMeans.
We obtain combinations of subsets of centroids, compute the standard deviation of 
a subset's coordinates along the $x$ and $y$ axes to determine which centroids are 
horizontally or vertically distributed, and classify centroids based on their 
relative positioning. 
Centroids are classified as either the event, or $+x$, $-x$, $+y$, or $-y$ mirror reflections. 
A set of possible event locations that spans the depth of the scintillator is 
computed using the event's centroid and \cref{eqn:mu}.
The number of mirror reflections, which reflections are present, and the 
initialization point for $\bm{p_0}$ are those that correspond to the 
highest value of $Q$ out of this set over 3, 4, and 5 clusters in KMeans.
The number of clusters corresponds to the number of mirror reflections plus the event.
This initialization procedure is described in \cref{alg:init}.
$\bm{\pi}$ is initialized to the uniform distribution.


\begin{algorithm}
\caption{\textbf{Initialization procedure.}} \label{alg:init}
\DontPrintSemicolon
\SetKwInOut{Parameter}{Arguments}
\SetKwFunction{KMeans}{KMeans}
\SetKwFunction{ClassifyCentroids}{ClassifyCentroids}
\SetKwFunction{PossibleEventLocations}{PossibleEventLocations}
\SetKwFunction{Q}{Q}
\SetKwComment{Comment}{// }{}
%\begin{algorithmic}
\Parameter{photon locations $\bm{t}$, photon weights $\bm{w}$, lens to sensor distance $S_2$}
\KwOut{event initialization location $\bm{p_{init}}$, number of mirror reflections $K_{out}$, mirror reflection classification $M_{out}$}
$Q_{max} = -\infty$\;
\For{K=3:5}{
  $\bm{\mu_k}^0 \leftarrow \KMeans(\bm{t}, \bm{w}, K)$\;
  $M \leftarrow \ClassifyCentroids(\bm{\mu_k}^0)$\;
  $S \leftarrow \PossibleEventLocations(\bm{\mu_{event}}^0, S_2)$ \Comment{\cref{eqn:mu}}
  \For{$\bm{\tilde{p_0}} \in S$}{
    $Q_{\bm{\tilde{p_0}}} \leftarrow \Q(\bm{\tilde{p_0}})$\;
    \If{$Q_{\bm{\tilde{p_0}}} > Q_{max}$}{
      $Q_{max} \leftarrow Q_{\bm{\tilde{p_0}}}$\;
      $\bm{p_{init}} \leftarrow \bm{\tilde{p_0}}$\;
      $M_{out} \leftarrow M$\;
      $K_{out} \leftarrow K-1$\;
    }
  }
}
\end{algorithm}

During the E-step, $r_{ik}$ is updated using \cref{eqn:r_ik} and the current 
values of $\bm{p_0}$ and $\bm{\pi}$.
During the M-step, $\bm{p_0}$ is updated by optimizing 
$Q$ with gradient ascent, 
and $\bm{\pi}$ is updated using $\pi_k=\frac{1}{N}\sum_{i=1}^N r_{ik}$.
Gradients are derived in Supplementary \cref*{sec:like,sec:weighted_like}.
In both the E and M steps, terms that correspond to a mirror reflection $k$ are 
included in the computation only if that mirror reflection is present in the image.
If truncation boundaries are known, then partial image truncations 
can be incorporated into the algorithm by zeroing $r_{ik}$ for photon
$i$ that lies in mirror $k$'s truncation zone.
We do not have accurate knowledge of truncation boundaries in experimental data,
so we only determine the presence of mirror reflections rather than partial 
truncations in the experiments.
The kaleidoscopic image model and event localization algorithm are validated on 
experimental data.


\subsection{Experimental data collection}
The experimental hardware consists of a SPAD array, lens, scintillator, and 1 $\mu$Ci Co-60 gamma-ray source.
We use the SPAD512 array (Pi Imaging) with microlenses for increased fill factor, 
which has $512 \times 512$ pixels and 16 $\mu$m pixel pitch.
The SPAD array is configured to capture 1-bit images with 1.5 $\mu$s 
integration time.
The lens is a 50 mm focal length Nikkor lens set to a f/1.2 aperture.
The scintillator is a GAGG(Ce)-HL crystal (Epic Crystal), which has a 150 ns decay 
constant, a 530 nm emission peak, and an index of refraction of 1.91.
Its geometry is a square pyramid with a 20 mm wide base and 5.77 mm 
height and a 120 degree opening angle at the apex.
Four surfaces of the scintillator are coated with enhanced specular reflector.
We account for the scintillator's index of refraction by assuming the 
scintillator's height is $5.77$ mm  / 1.91 $=3.02$ mm and compute apparent event 
locations in the smaller volume.
The camera's lateral field of view (FOV) covers approximately $5 \times 5$ mm at 
the focal plane.
The scintillator is positioned such that its apex is in-focus and centered in the 
FOV, and that its corners are aligned with the corners of the FOV. 
There is an air gap of approximately 30 mm between the scintillator and the lens.
The entire setup is placed inside a light-tight box to keep ambient light out.
The scintillator is shown in Supplementary Fig. \ref*{fig:scintillator}.
The experimental camera focus on the scintillator is shown in Supplementary Fig. \ref*{fig:experiment_focus}. 

Data collection took place at about 21 degrees Celsius ambient temperature.
We first captured 130,000 images without the gamma-ray source to characterize the 
dark count rate and zero out 5\% of pixels with the highest dark count rates.
After zeroing 5\% of pixels, we observe a median of 4 dark counts per image.
A histogram of dark counts per image is shown in Supplementary Fig. \ref*{fig:dark_counts_hist}.

We collect data by placing the gamma-ray source adjacent to the scintillator's 
apex and passively capturing images.
All computations are performed in post-processing.
Zeroing 5\% of pixels is applied to all images.
We capture 13,000,000 images with the gamma-ray source present and 
discard images with less than 60 counts.
Histograms of counts in an image are shown in Supplementary \cref*{fig:cap_counts_hist}.


\subsection{Algorithm parameter values and stopping criteria}
We use $q=10$ nearest neighbors and $\nu=10$ pixels for assigning photon weights 
using \cref{eqn:weights}.
Up to ten possible event locations are evaluated in the initialization procedure 
equispaced over the scintillator's depth.
During one M-step, we run gradient ascent for 1,000 steps with a step size of 1e-7.
We run the EM algorithm until the distance in the estimated event location between 
consecutive steps is less than 0.01 mm, or until 100 steps are taken.


\subsection{Experimental camera parameter calibration}
Experimental camera parameter values for $S_1$, $S_2$, and $a$ are calibrated by 
optimizing $Q$ in a grid search on a chosen experimental image.
We choose the image shown in \cref{fig:calibration}, where the event is 
approximately centered with evident mirror reflections, and we manually remove 
dark counts.
The grid search is performed over event locations at $x$-$y$ coordinates 
$(0,0)$ mm and $z$-coordinates that span the depth of the scintillator.
This resulted in $S_1=37.24$ mm, $S_2=109.48$ mm, and $a=0.056$ at event location 
(0, 0, 2.61) mm.

\ifthenelse{\boolean{figs_in_text}}{

\begin{figure*}
\centering
\includegraphics[width=\linewidth]{calibration.pdf}
\caption{\textbf{Experimental calibration image.} a) The original image. b) The image after manually 
removing dark counts overlaid with the Gaussian components found during the 
calibration procedure.
Each dashed red circle is centered on the Gaussian component's mean. 
The inner and outer circles are one and two standard deviations in radius, respectively.
Pixels with a photon are enlarged with a $3 \times 3$ filter for visualization purposes.
} 
\label{fig:calibration}
\end{figure*}
}{}

\subsection{Light collection and truncation theory validation}
We validate the theory on kaleidoscopic light collection and image truncations by 
observing that the derived truncation lines align with photon arrivals in a 
simulated image.
We simulate the image of an event in a kaleidoscopic scintillator in the shape of 
a square pyramid using a thin lens and ray tracing. 
The kaleidoscope is 3.02 mm in height and has a 20 mm base length.
We set its index of refraction to 1 and place its apex at $z_w=0$ mm.
A thin lens with 30 mm focal length and 20 mm diameter is placed at $z_w=50$ mm.
A $512 \times 512$ sensor with 16 $\mu$m pixel pitch is placed at $z_w=125$ mm.
100,000 photons are emitted isotropically from $(0.25, 1, 2)$ mm (world 
coordinates).
We emit an unrealistically high number of photons so that image truncations are 
clearly observable.
Acceptance zones for each mirror reflection are derived and overlaid on the image.
The resulting image is shown in \cref{fig:trunc_examples}.

\subsection{Kaleidoscopic model validation}
We select six experimental images and overlay the algorithm's estimated Gaussian 
components (\cref{fig:example_figures}) to validate the presence of mirror 
reflections in accordance with the kaleidoscopic model.
Due to non-idealities, truncation zone boundaries in experimental images do not 
exactly match those derived theoretically using a thin lens.
We do not attempt to derive truncation lines in experimental images.
Rather, truncations are evident from missing mirror reflections in the selected images.

\subsection{Algorithm validation}
Experimental events cannot be controlled, so their ground truth locations are unknown.
Therefore, to validate the algorithm is locating the event, we report agreement of 
multiple measurements of the event's location as follows.
We use experimental images that contain the event and four mirror reflections as 
test images.
The number of mirror reflections in an image is determined using the algorithm.
Then, we create new images of the event by removing combinations of one or two 
mirror reflections from the image.
Photon $i$ is classified as belonging to mirror reflection $k$ according to 
$\max_k r_{ik}$ using $r_{ik}$ values obtained from running the algorithm on the 
original test image with four mirror reflections.
We obtain three groups of images per test image.
The first group contains the original image, each combination of single mirror 
reflection removals, and each combination of double mirror reflection removals for 
a total of 11 images per test image.
The second group contains the original image and each combination of single mirror 
reflection removals for a total of 5 images per test image.
The third group contains the original image and each combination of double mirror 
reflection removals for a total of 7 images per test image.
We run the algorithm to estimate the event's location using each image.
In each group, we compute the mean estimated event location and record the 
distance between the mean location and the estimated location for each image.
The distribution of this distance is used to report the agreement in event 
location measurements.
Selecting images that contain four mirror reflections and at least 60 counts 
resulted in 2,251 test images.

We test for the algorithm's convergence by using two different initialization 
methods for $\bm{p_0}$. 
We either initialize $\bm{p_0}$ adaptively to the image as in the initialization 
procedure in \cref{alg:init} or to $(0,0,1.5)$ mm.
We refer to the two initializations as ``regular" and ``fixed", respectively.
The test images for this convergence experiment consist of all experimental images 
that contain at least 60 counts, for a total of 4,351 images.

Results for these two experiments are reported in the ``Results" section.





\bibliographystyle{unsrt}
\bibliography{references.bib}




\section{Acknowledgments}
\noindent
U.S. Department of Energy (Disclaimer): 
This work was prepared as an account of work sponsored by an agency of the United 
States Government. 
Neither the United States Government nor any agency thereof, nor any of their 
employees, nor any of their contractors, subcontractors or their employees, makes 
any warranty, express or implied, or assumes any legal liability or 
responsibility for the accuracy, completeness, or any third parties use or the 
results of such use of any information, apparatus, product, or process disclosed, 
or represents that its use would not infringe privately owned rights. 
Reference herein to any specific commercial product, process, or service by trade 
name, trademark, manufacturer, or otherwise, does not necessarily constitute or 
imply its endorsement, recommendation, or favoring by the United States 
Government or any agency thereof or its contractors or subcontractors. 
The views and opinions of authors expressed herein do not necessarily state or 
reflect those of the United States Government or any agency thereof, its 
contractors or subcontractors.


\section{Author contributions}
A.B. and A.V. conceived and developed the methods. 
A.B. carried out the experiments. 
A.B. wrote the manuscript with support from A.V..
J.M. developed the FPGA circuitry for actuating and reading from the SPAD sensor. 
D.A. helped modify the light-tight box and provided consultation for setting up and operating the camera.
A.V. acquired funding and provided supervision.


\section{Competing interests}
The authors declare no competing interests.




\newpage

\ifthenelse{\not\boolean{figs_in_text}}{

\begin{figure*}
\centering
\includegraphics[width=\linewidth]{teaser_pipeline.jpg}
\caption{\textbf{Overview of image capture and event localization method.}
An image is composed of light that is emitted from the scintillation event and 
reaches the camera either directly or after reflecting off mirrors of a 
kaleidoscopic scintillator.
The locations of an event's mirror reflections are known for a given event 
location and kaleidoscope geometry.
This relationship between the event and mirror reflections is embedded in a 
Gaussian mixture model whose likelihood is maximized to estimate the event's location.
} 
\label{fig:teaser}
\end{figure*}

\begin{figure}
\centering
\includegraphics[width=\linewidth]{trunc_examples.pdf}
\caption{\textbf{Simulated kaleidoscopic image with theoretical acceptance zones.} 
Acceptance zones are derived for each mirror reflection and overlaid on the image 
for the (a) +$x$, (b) +$y$, (c) -$x$, and (d) -$y$ mirror reflections.} 
\label{fig:trunc_examples}
\end{figure}


\begin{figure*}
\centering
\includegraphics[width=\linewidth]{example_figures.pdf}
\caption{\textbf{Selected experimental images.} Experimental images overlaid with the algorithm's estimated Gaussians.
Each dashed red circle is centered on the Gaussian component's mean. 
The inner and outer circles are one and two standard deviations in radius, respectively.
Pixels with a photon are enlarged with a $3 \times 3$ filter for visualization purposes.
} 
\label{fig:example_figures}
\end{figure*}



\begin{figure*}
\centering
\includegraphics[width=\linewidth]{crossval_error.pdf}
\caption{\textbf{Agreement in event location measurements.} Histogram of distances between mean estimated event location and each image's estimated event location after mirror reflection removals. 
(a) Distances including both single and double mirror reflection removals. Mean $\pm$ stdev: 0.24 $\pm$ 0.34 mm, 24,761 distances.
(b) Distances including single mirror reflection removal. Mean $\pm$ stdev: 0.16 $\pm$ 0.28 mm, 11,255 distances.
(c) Distances including double mirror reflection removal. Mean $\pm$ stdev: 0.27 $\pm$ 0.35 mm, 15,757 distances.
} 
\label{fig:crossval_error}
\end{figure*}


\begin{figure*}
\centering
\includegraphics[width=\linewidth]{fixed_init_results.pdf}
\caption{\textbf{Optimization convergence.}
(a) Distances in estimated event locations between the regular and fixed initialization methods. Mean $\pm$ stdev: $0.38 \pm 0.80$ mm, 4,351 events.
(b) Nummber of steps taken in the EM algorithm from the regular initialization point. Mean $\pm$ stdev: $9.3 \pm 14.2$ steps, 4,351 events.
(c) Nummber of steps taken in the EM algorithm from the fixed initialization point. Mean $\pm$ stdev: $13.5 \pm 14.9$ steps, 4,351 events.
} 
\label{fig:convergence}
\end{figure*}


\begin{figure}
\centering
\includegraphics[width=.33\linewidth]{optical_config.pdf}
\caption{\textbf{Imaging parameters and coordinate systems.} 
The optical configuration consists of a lens with diameter $A$, distance from 
lens to focal plane $S_1$, and distance from lens to sensor $S_2$. The diameter 
of an event's image on the sensor is $c$. The camera is focused at the 
scintillator's apex where the world coordinate system origin is located. The 
camera coordinate system origin is located at the lens. The figure only shows 
light emitted directly to the camera.} 
\label{fig:optical_config}
\end{figure}


\begin{figure*}
\centering
\includegraphics[width=\linewidth]{trunc_theory.pdf}
\caption{\textbf{Mirror apertures and image truncations.} 
An event emits light onto a mirror that reflects into the camera. 
Some light might not reach the sensor due to finite mirrors and defocus blur.
(a) The mirror reflection is located beyond the focal plane. 
All light that forms the mirror reflection reaches the sensor.
(b) The mirror reflection is located beyond the focal plane. 
Some light from the mirror reflection is stopped at the mirror's edge and 
truncated on the sensor.
(c) The mirror reflection is located within the focal plane.
All light that forms the mirror reflection reaches the sensor.
(d) The mirror reflection is located within the focal plane. 
Some light from the mirror reflection is stopped at the mirror's edge and 
truncated on the sensor.
(e) Light from a double mirror reflection is stopped at both mirrors' edges and truncated on the sensor. 
The mirror for the second reflection is illustrated along the optical axis.} 
\label{fig:trunc_theory}
\end{figure*}


\begin{figure}
\centering
\includegraphics[width=.4\linewidth]{trunc_teaser.jpg}
\caption{\textbf{Image truncation example in 3D.}
The image of an event and one mirror reflection is shown. 
Light corresponding to the mirror reflection is truncated at the mirror's edge, 
resulting in a truncation line in the image.
The truncation line only applies to that mirror reflection.}
\label{fig:trunc_teaser}
\end{figure}  





\begin{figure*}
\centering
\includegraphics[width=\linewidth]{calibration.pdf}
\caption{\textbf{Experimental calibration image.} a) The original image. b) The image after manually 
removing dark counts overlaid with the Gaussian components found during the 
calibration procedure.
Each dashed red circle is centered on the Gaussian component's mean. 
The inner and outer circles are one and two standard deviations in radius, respectively.
Pixels with a photon are enlarged with a $3 \times 3$ filter for visualization purposes.
} 
\label{fig:calibration}
\end{figure*}


\begin{table}[h!]
\centering
\begin{tabular}{|c|ccccc|}
\hline
Counts in a test image & 60 - 79 & 80 - 99 & 100 - 119 & 120 - 139 & 140 - 159 \\
\hline
Mean (mm)                & 0.28 & 0.26 & 0.19 & 0.19 & 0.15 \\
Standard dev.  (mm)      & 0.36 & 0.35 & 0.30 & 0.29 & 0.21  \\
Number of distances    & 9,317  & 7,524  & 4,774  & 2,354  & 792 \\
\hline
\end{tabular}
\caption{\textbf{Agreement in event location measurements.} Mean and standard deviation of distances between mean estimated 
event location and each image's estimated event location after both single and 
double mirror reflection removals reported over different ranges of counts in a 
test image.}
\label{tab:crossval_thresh_error}
\end{table}

  
}{}




\end{document}
